\documentclass[11pt]{article}
\usepackage{exercise}

\usepackage{amsmath,times,color,url}
\usepackage{amsfonts,bussproofs}

\usepackage{alltt}
\usepackage{proof}
\usepackage{tikz}
\usepackage{cancel}
\usepackage{mathtools}



\def\set#1{\{#1\}}
\def\Set#1{\{#1\}}
\def\z{{\tt 0}}
\def\o{{\tt 1}}
\def\t{{\tt 2}}
\def\a{{\tt a}}
\def\b{{\tt b}}
\def\c{{\tt c}}
\def\d{{\tt d}}
\def\DFA{{\sf DFA}}
\def\NFA{{\sf NFA}}
\def\R{{\cal R}}
\def\Picture#1{\strut\centerline{\input #1.eepic}}


\definecolor{midnightblue}{cmyk}{0.98,0.13,0,0.43}
\definecolor{olivegreen}{cmyk}{0.64,0,0.95,0.40}
\definecolor{rawsienna}{cmyk}{0,0.72,1,0.45}
\definecolor{fuchsia}{cmyk}{0.47,0.91,0,0.08}
\definecolor{shaded}{rgb}{0.7, 0.75, 0.71}

\newcommand{\Blue}[1]{{\color{blue}#1}}
\newcommand{\Red}[1]{{\color{red}#1}}
\newcommand{\Brown}[1]{{\color{brown}#1}}
\newcommand{\OliveGreen}[1]{{\color{olivegreen}#1}}
\newcommand{\RawSienna}[1]{{\color{rawsienna}#1}}
\newcommand{\MidnightBlue}[1]{{\color{midnightblue}#1}}
\newcommand{\Fuchsia}[1]{{\color{fuchsia}#1}}
\newcommand{\Gray}[1]{{\color{shaded}#1}}

\usepackage{prftree}


\newcommand{\M}[1]{{\color{midnightblue}#1}}
\newcommand{\notl}{\neg}
\newcommand{\orl}{\vee}
\newcommand{\andl}{\wedge}
\newcommand\impl{\rightarrow}

\newcommand{\setof}[1]{\{#1\}}
\newcommand{\true}{1}
\newcommand{\false}{0}

\newcommand{\Th}{\mathcal{T}}

\newcommand{\BRr}{\mathrm{B}\mathrm{R}}   
\newcommand{\Sup}{\mathrm{S}\mathrm{up}}   
\newcommand{\SUPiss}{\mathrm{S}\mathrm{up}_{\succ,\sel}}   % SUPis with selection
\newcommand{\sel}{\sigma}


% object level equality
\newcommand{\oeq}{=}%\approx}
\newcommand{\oneq}{\neq}%\not\approx}

\newcommand{\SUPis}{\mathbb{S}\mathrm{up}}   % SUP
\newcommand{\Iis}{\mathbb{I}} %inference system


\newcommand{\BRis}{\mathbb{BR}}             % binary resolution inference system
%\newcommand{\Solution}[1]{#1}
\newcommand{\Solution}[1]{}

\newcommand{\Union}{\bigcup}


\begin{document}
\problemset{November 5, 2025}


\section*{\exercises{8}}



\problem{8.1}  Let $f$ be a unary and $g$ be a binary function symbol.
Further let $a, b, c$ be constants, and $x, y, z$ be variables.
We define the weight function $w(s) = w(v) = 1$, for every symbol $s$
and variable $v$, and let $g \gg f \gg c \gg b \gg a$.
Answer the following questions using a KBO with the weight
function $w$ and the precedence relation $\gg$ to order terms,
and its extension to compare literals and clauses. 

\begin{itemize}
  \item[(a)] Do the clauses $C_1$ and $C_2$ make the clause $C_3$ redundant?
    \begin{align*}
      C_1:~ &~ a \oneq b \lor f(a) \oneq a \\
      C_2: ~ &~ f(f(x)) \oeq a \\
      C_3: ~ &~ f(f(b)) \oneq b \lor f(a) \oneq a
    \end{align*}
    % Answer: no, because f(f(x)) = a is incomparable to f(f(b)) != b \/ f(a) != a
    %         shouldn't the answer be "yes"? because we can use demodulation(C_2, C_3) to derive C_1.
  \item[(b)] Does the clause $ C_4 $ make the clause $ C_5 $ redundant?
    \begin{align*}
      C_4: ~ &~ f(g(x, a)) \oneq f(y) \\
      C_5: ~ &~ f(g(x, z)) \oneq f(g(y, b)) \lor f(g(x, b)) \oneq f(g(a, b))
    \end{align*}
    % Answer: yes because f(g(x,a)) != f(y) is unsat, and smaller than the latter clause
  \item[(c)] Does the clause $ C_6 $ make the clause $ C_7 $ redundant?
    \begin{align*}
      C_6: ~ &~ g(x, y) \oneq f(x) \\
      C_7: ~ &~ g(f(x), f(z)) \oneq f(f(x)) \lor g(a, b) \oneq c
    \end{align*}
    % Answer: yes because f(g(x,a)) != f(y) < g(x, f(z)) != f(f(x)) < g(x, f(z)) != f(f(x)) \/ g(a,b) != c
\end{itemize}


\medskip

\Solution{
  \noindent{\bf Solution.}


\begin{itemize}
  \item[(a)] Yes.
  Recall, that to show that $C_3$ is redundant w.r.t. $\{C_1,C_2\}$,
  we need to show that every ground instance of $C_3$ is redundant
  w.r.t. the set $C_{1,2}^*$ of all ground instances of
  $\{C_1,C_2\}$.\\
  
  
Note that $C_3$ is ground.\\


Using the substitution $x\mapsto b$, the instance $f(f(b)) = a$ of
$C_2$ is in $C_{1,2}^*$.
As $C_1$ is ground,  $C_1$ is also in $C_{1,2}^*$.
Moreover, the ground clauses $\{f(f(b)) = a, C_1\}$ imply $C_3$, and we also
have that $C_3 \succ (f(f(b)) = a)$ and $C_3 \succ C_1$.
Hence, every ground instance of $C_3$ (that is, $C_3$) is redundant
w.r.t. $\{f(f(b)) = a, C_1\}\subseteq C_{1,2}^*$
This concludes, that the clauses $C_1,C_2$ make $C_3$
redundant.
\bigskip



  \item[(b)] Yes, because $C_4 : f(g(x, a)) \neq f(y)$ is unsat, and
  smaller than $C_5$.\\
  

In more detail, we proceed as in point (a) and show that every ground instance of $C_5$ is redundant
w.r.t. the set $C_4^*$ of all ground instances of $C_4$. Note that $C_4^*$ contains an unsatisfiable ground
instance: 

\[D_1 : f(g(a,a)) \neq f(g(a,a))\]

Since $D_1$ unsatisfiable, it implies every ground instance of $C_5$
(and also $C_5$).

It remains to prove that every ground instance of $C_5$ is bigger than
$D_1$. The smallest ground instance of $C_5$ is:

\[D_2:   f(g(a,a)) \neq f(g(a,b)) \orl f(g(a,b)) != f(g(a,b))\]


which is bigger than $D_1$.

Hence, every ground instance of $C_5$ is
redundant w.r.t. $D_1\subseteq C_4^*$.

As such, $C_4$ makes $C_5$
redundant. 


\bigskip

\item[(c)] Yes.

In short,  because firstly we can rename the variable $z$ to $y$ in the second clause.  
    This means, since $g(x, y) \oneq f(x) \prec g(f(x), f(z)) \oneq f(f(x))$, then $C_6 \prec C_7$.
    Further $g(f(x), f(z)) \oneq f(f(x))$ is an instance of $g(x, y)
    \oneq f(x)$, which means that $C_6$ implies $C_7$.\\


In more detail, we proceed as in points (a)-(b):
we show that every ground instance of $C_7$ is redundant
w.r.t. the set $C_6^*$ of all ground instances of $C_6$. 

Take any ground instance of $C_7$. It has the form: 

\[  D_1:  g(f(s),f(t)) \neq  f(f(s)) \vee  g(a,b) \neq c\]

where $s,t$ are gound terms. Take the following ground instance of $C_6$:

\[  D_2 : g(f(s),f(t)) \neq f(f(s))\]


We have $D_2 \models D_1$ and $D_1\succ D_2$. Hence, $C_7$ is redundant.



\end{itemize}
  }


%redundancy, subsumption demodulation 
\problem{8.2}
Consider the following inference:
%Aduct21S#27
\[\prftree{
  x \oeq f(c)\lor p(x)
}{
  f(h(b)) \oeq h(g(y,y)) \lor h(g(d,b))\oneq f(c)
}{
  p ( h ( g ( d , b ) ) ) \lor f ( h ( b) )  \oeq  h(g(y,y))
}\]
%in the non-ground
%superposition inference system $\SUPis$ ({\bf without} the rules of the non-ground
%binary resolution inference system $\BRis$),
where $p$ is a predicate symbol, $f$, $g$, $h$ are function symbols,
$b, c, d$ are constants, and $x$, $y$ are variables.
%
\begin{itemize}
    \item[(a)]
        Prove that the above inference is a sound inference.
        % suppose M |=  x = f(c) \/ p(x)                             (1)
        %         M |=  f(h(b)) = h(g(y,y)) \/ h(g(d,b)) != f(c)     (2)
        % If M |= f(h(b)) = h(g(y,y)), then the conclusion is obviously true in M.
        % othersise due to (2) it must be the case that M |= h(g(d,b)) != f(c), which means due to (1) that M |= p(g(d,b)) must hold, which means that the conclusion is true in M as well.
    \item[(b)]
        Is the above inference a simplifying inference of the
        non-ground superposition system $\SUPis$,
        for any KBO?
        Justify your answer. 

        % Countermodel for Hyp1, Concl |= Hyp2
        % M |= p(x)
        % M(f)(C) = H
        % M(f)(..) = F
        % M(h(...)) = H
        % M(g(...)) = G
        % M(c) = C
        % M(d) = D
        % M(b) = B

        % Countermodel for Hyp2, Concl |= Hyp1
        % M |= ~p(x)
        % domain = {A, B}
        % M(f)(..) = A
        % M(h)(..) = A

\end{itemize}

%
\medskip

\Solution{
  \noindent{\bf Solution.} 
\begin{itemize}
    \item[(a)]
        Let $M$ be a model of both assumptions of the rule: 
        % suppose M |=  x = f(c) \/ p(x)                             (1)
        %         M |=  f(h(b)) = h(g(y,y)) \/ h(g(d,b)) != f(c)     (2)
        \begin{align*}
          M &\models x = f(c) \lor p(x) & \text{(1)} \\
          M &\models f(h(b)) = h(g(y,y)) \lor h(g(d,b)) \neq f(c) & \text{(2)}\\
        \end{align*}
        % If M |= f(h(b)) = h(g(y,y)), then the conclusion is obviously true in M.
        If $M \models f(h(b)) = h(g(y,y))$, then the conclusion is obviously true in $M$.
        % othersise due to (2) it must be the case that M |= h(g(d,b)) != f(c), which means due to (1) that M |= p(g(d,b)) must hold, which means that the conclusion is true in M as well.
        Otherwise due to (2) we know that it must be the case that $M \models h(g(d,b)) \neq f(c)$. 
        This means due to (1) that $M \models p(g(d,b))$ must hold, which means that the conclusion is true in this case as well.
    \item[(b)]
        We will call the left assumption $C_1$ and the right assumption of the rule $C_2$, and the conclusion of the rule $D$.

        In order for the rule to be a simplifying rule, it could either make $C_1$, or $C_2$ redundant, so there are two cases to check.

        Let's first have a look at whether $C_1$ is being made redundant. 
        In order for that to hold we would need to have that $C_2, D \models C_1$. 
        This does not hold since the model $M_1$ is a counterexample.

        % Countermodel for Hyp2, Concl |= Hyp1
        % M |= ~p(x)
        % domain = {A, B}
        % M(f)(..) = A
        % M(h)(..) = A
        \newcommand\A{\mathsf a}
        \newcommand\B{\mathsf b}
        \newcommand\C{\mathsf c}
        \newcommand\D{\mathsf d}
        \begin{align*}
          M_1(\top) &= \Set{\A, \B}\\
          M_1(p) &= \emptyset\\
          M_1(f)(x) &= \A\\
          M_1(h)(x) &= \A\\
          M_1(g)(x,y) &= \A\\
        \end{align*}
        (Note that by $M(\top)$ we denote the domain of the model here.)

        Similarly we can build a counterexample $M_2$ for the statement $C_1, D \models C_2$.

        % Countermodel for Hyp1, Concl |= Hyp2
        % M |= p(x)
        % M(f)(C) = H
        % M(f)(..) = F
        % M(h(...)) = H
        % M(g(...)) = G
        % M(c) = C
        % M(d) = D
        % M(b) = B
        \newcommand\F{\mathsf f}
        \newcommand\G{\mathsf g}
        \renewcommand\H{\mathsf h}
        \begin{align*}
          M_2(p) = M_2(\top) &= \Set{\B, \C, \D, \F, \G, \H}\\
          M_2(b) &= \B\\
          M_2(c) &= \C\\
          M_2(d) &= \D\\
          M_2(f)(x) &= \begin{cases}
            \H & \text{if $x = \C$}\\
            \F & \text{else}
          \end{cases}\\
          M_2(g)(x, y) &= \G\\
          M_2(h)(x) &= \H\\
        \end{align*}


\end{itemize}


}


\problem{8.3} Let $\succ$ be a Knuth-Bendix ordering  KBO and
$\sigma$ a well-behaved selection function wrt $\succ$.
Consider the following inference instance of the non-ground
superposition calculus
$\textrm{Sup}_{\succ,\sigma}$: 
\[
\begin{array}{c}
l=r   \qquad l'=t \ \orl\ D\\
\hline\\[-.75em]
(r\theta=t) \orl D
\end{array}
\]
%
where $l\theta=l'$, 
%\begin{itemize}
%\item
  $ l\succ r$, $l'\succ t$, 
%\item
  %$\theta=mgu(l,l')$,
          $r\theta \succ t$, $(l=r)\theta \succ D$, $(l'=t)\succ D$ and  $l' = t$ is selected
  in $l'=t\orl D$. 
          % \end{itemize}

          \begin{itemize}
            \item[(a)] Is $\big((r\theta=t) \orl D\big)\ \succ\  \big(l'=t \
              \orl\ D\big)$?

              \medskip
\Solution{              
\noindent
{\bf Solution:} No. 

As $\succ$ is a simplification ordering, from $l\succ r$ we obtain
$l\theta \succ r\theta$, for any substitution $\theta$, and further, using
$l\theta = l'$, we get $l'\succ r\theta$. Then, by
the atom/literal/bag extension properties of $\succ$, we have $l'=t
\succ r\theta=t$, and hence $l'=t \orl D
\succ r\theta=t \orl D$.

}

              
            \item[(b)]
              Does $l' =t\ \orl\ D$ always become redundant after the inference is
              applied?

              \medskip
              \Solution{
\noindent
{\bf Solution:}
No.

Recall that a formula is redundant iff it is the logical consequence of smaller formulas in the search space.
More precisely in the non-ground case,
considering that a non-ground formula represents all its ground instances:
a non-ground formula $\varphi$ is redundant in $S$ if
every ground instance $\varphi\sigma$ is a logical consequence
of ground instances of formulas from $S$ that are smaller that $\varphi\sigma$.
%
% % The logical consequence $l=r, \, r\theta=t \lor D \models l'=t \lor D$
% This means we would have to prove the following properties:
% \begin{enumerate}
%     \item
%         $l=r, \, r\theta=t \lor D \models l'=t \lor D$
%     \item
%         $l'=t \lor D \succ l=r$
%     \item
%         $l'=t \lor D \succ r\theta=t \lor D$
% \end{enumerate}

In particular, one of the statements to be proved is
that for each ground substitution $\sigma$
there exists some ground substitution $\sigma'$
such that
$(l'=t \lor D)\sigma \succ (l=r)\sigma'$.

It is easy to see that this cannot always be true if we consider an instance of
the inference rule where $l$, $r$, and $t$ are ground terms.
In particular, this means we have $l = l'$
and thus $l=r \succ l'=t$ (because of the fourth side condition $r\theta \succ t$).
Furthermore, from the fifth side condition we also get $l=r \succ D$.
By properties of the bag extension of orderings, this means
$l=r \succ l'=t \lor D$, which is exactly the opposite of the required ordering constraint.

% Consider the following counter-example
% \[
% \begin{array}{c}
%     f(f(c))=f(c)   \qquad f(f(c))=c \ \orl\ f(c)\neq d \\
% \hline\\[-.75em]
% f(c)=c \orl f(c)\neq d
% \end{array}
% \]
% where $f$ is a unary function symbol and $c,d$ are constant symbols.
% It is a ground instance of the above rule with $l = f(f(c))$, $r = f(c)$, $t = c$, $D = f(c) \neq d$,
% and $\theta$ the identity substitution.
% Consider a KBO ordering with weights $w(f) = w(c) = w(d) = 1$ and precedence $f \gg c \gg d$.
% With this ordering, the side conditions of the rule are satisfied.
}
\end{itemize}
%\noindent Justify your answers.


\problem{8.4} Consider the following inference: 
\[
  \begin{array}{c}
    f(a,b)\neq g(x)  \orl f(x,y)=g(a) \qquad f(y,b)=g(y) \orl
    f(x,a)=g(x) \orl f(y,y)= g(z)
\\
\hline\\[-.75em]
f(a,a)=g(a)
\end{array}
\]
in the non-ground
superposition inference system $\SUPis$ (including the rules of the non-ground
binary resolution  inference system $\BRis$),  
where $f,g$ are function symbols, %$p$ is a predicate symbol, $f$ is a function symbol,
$x,y,z$ are variables, and $a,b$ are
constants.  %\\



          \begin{itemize}
            \item[(a)] Prove that the above inference is a sound
              inference of $\SUPis$.
            \item[(b)]  Is the above inference a simplifying inference of the
        non-ground superposition system $\SUPis$,
        for any KBO? 
              Justify your answer.
              \end{itemize}

\medskip

\Solution{
{\bf Solution:}\medskip\\
(a) We use sound inferences of $\SUPis$ to show that conclusion of the above inference follows from its premises. The following derivation uses the unifiers $\theta:=\{x\mapsto a,y\mapsto a\}$ and $\eta:=\{z\mapsto a\}$:
\begin{prooftree}
\AxiomC{$\underline{f(a,b)\neq g(x)} \lor f(x,y)=g(a)$}
\AxiomC{$\underline{f(y,b)=g(y)} \lor f(x,a)=g(x) \lor f(y,y)= g(z)$}
\LeftLabel{$\theta$}
\RightLabel{(BR)}
\BinaryInfC{$f(a,a)=g(a)\lor f(a,a)=g(a) \lor f(a,a)= g(z)$}
\RightLabel{EF}
\UnaryInfC{$f(a,a)=g(a)\lor g(a)\neq g(a) \lor f(a,a)= g(z)$}
\RightLabel{EF}
\LeftLabel{$\eta$}
\UnaryInfC{$f(a,a)=g(a)\lor g(a)\neq g(a) \lor g(a)\neq g(a)$}
\RightLabel{ER}
\UnaryInfC{$f(a,a)=g(a)\lor g(a)\neq g(a)$}
\RightLabel{ER}
\UnaryInfC{$f(a,a)=g(a)$}
\end{prooftree}

(b) 

 %(b):


In order for the inference to simplify either of its premises, they must be made redundant by the conclusion and the other premise.
Recall that for a non-ground clause $C$ to be made redundant by clauses $C_i$, we need to have that for every ground instance $C^\theta$ of $C$, there are ground instances $C_i^\theta$ of $C_i$, such that these $C_i^\theta$ make $C^\theta$ redundant.
 %non-ground superposition, completeneess
We will show that this is not the case by showing that there are counter-models to the redundancy.
 
 % first premise not redundant. 
 % Consider interpretation I with domain 
 % Domain D = {a,b} 
 % I(f(x, a)) = x
 % I(f(x, b)) = b
 % I(g(x)) = x
 % where have u
 %           I |= f(a, a) = g(a)
 %           I |= f(x, a) = g(x)
 %  but not  I |= f(a,b) != g(x) \/ f(x,y) = g(a)
 %           for x -> b, y -> b

Let's first have a look at whether the left premise is redundant.
Consider the interpretation $I$ with domain $\{a, b\}$.
We define
\begin{align*}
I(f)(x, a) = x\\
I(f)(x, b) = b\\
I(g)(x) = x\\
\end{align*}

Then we have $I$ satisfies the conclusion $f(a,a) = a$.
Further  $I \models f(x, a) = g(x)$, hence it satisfies the right premise,
but it does not satisfy the left premise for the grounding $\{x \mapsto b, y \mapsto b\}$.
Therefore the left premise cannot be redundant.

\medskip 
 
 % second premise not redundant. 
 % Consider interpretation I with domain 
 % Domain D = {a, b} 
 % I(f(x, y)) = a
 % I(g(x)) = x
 % where have u
 %           I |= f(a, a) = g(a)
 %           I |= f(x, y) = g(a)
 %  but not  I |= f(y, b) = g(y) | f(x, y) = g(y) | f(y,y) = g(z)
 %           for y -> b, z -> b, x -> b 
 
Let us now show that the right premise is not redundant.
For that consider the interpretation $I$ with the same domain $\{a, b\}$.
 
\begin{align*}
I(f)(x, y) = a\\
I(g)(x) = x\\
\end{align*}

Again this interpretation satisfies the conclusion, and as it satisfies $f(x,y) = g(a)$, it also satisfies the left premise.
Further it does not satisfy the rightpremise for the grounding $\{ x \mapsto b, y \mapsto b, z \mapsto b\}$.
Therefore the right premise is not redundant either.

}
 




\end{document}

