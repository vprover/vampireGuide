\documentclass[11pt]{article}
\usepackage{exercise}

\usepackage{amsmath,times,color,url}
\usepackage{amsfonts,bussproofs}

\usepackage{alltt}
\usepackage{proof}
\usepackage{tikz}
\usepackage{cancel}
\usepackage{mathtools}



\def\set#1{\{#1\}}
\def\Set#1{\{#1\}}
\def\z{{\tt 0}}
\def\o{{\tt 1}}
\def\t{{\tt 2}}
\def\a{{\tt a}}
\def\b{{\tt b}}
\def\c{{\tt c}}
\def\d{{\tt d}}
\def\DFA{{\sf DFA}}
\def\NFA{{\sf NFA}}
\def\R{{\cal R}}
\def\Picture#1{\strut\centerline{\input #1.eepic}}


\definecolor{midnightblue}{cmyk}{0.98,0.13,0,0.43}
\definecolor{olivegreen}{cmyk}{0.64,0,0.95,0.40}
\definecolor{rawsienna}{cmyk}{0,0.72,1,0.45}
\definecolor{fuchsia}{cmyk}{0.47,0.91,0,0.08}
\definecolor{shaded}{rgb}{0.7, 0.75, 0.71}

\newcommand{\Blue}[1]{{\color{blue}#1}}
\newcommand{\Red}[1]{{\color{red}#1}}
\newcommand{\Brown}[1]{{\color{brown}#1}}
\newcommand{\OliveGreen}[1]{{\color{olivegreen}#1}}
\newcommand{\RawSienna}[1]{{\color{rawsienna}#1}}
\newcommand{\MidnightBlue}[1]{{\color{midnightblue}#1}}
\newcommand{\Fuchsia}[1]{{\color{fuchsia}#1}}
\newcommand{\Gray}[1]{{\color{shaded}#1}}

\usepackage{prftree}


\newcommand{\M}[1]{{\color{midnightblue}#1}}
\newcommand{\notl}{\neg}
\newcommand{\orl}{\vee}
\newcommand{\andl}{\wedge}
\newcommand\impl{\rightarrow}

\newcommand{\setof}[1]{\{#1\}}
\newcommand{\true}{1}
\newcommand{\false}{0}

\newcommand{\Th}{\mathcal{T}}

\newcommand{\BRr}{\mathrm{B}\mathrm{R}}   
\newcommand{\Sup}{\mathrm{S}\mathrm{up}}   
\newcommand{\SUPiss}{\mathrm{S}\mathrm{up}_{\succ,\sel}}   % SUPis with selection
\newcommand{\sel}{\sigma}


% object level equality
\newcommand{\oeq}{=}%\approx}
\newcommand{\oneq}{\neq}%\not\approx}

\newcommand{\SUPis}{\mathbb{S}\mathrm{up}}   % SUP
\newcommand{\Iis}{\mathbb{I}} %inference system


\newcommand{\BRis}{\mathbb{BR}}             % binary resolution inference system
%\newcommand{\Solution}[1]{#1}
 \newcommand{\Solution}[1]{}

\newcommand{\Union}{\bigcup}


\begin{document}
\problemset{Lecture 2, October 6, 2025}

%%%% Lecture 2 - Exercises 2

\section*{\exercises{2}}

  
\problem{2.1}
Let $S$ be the following set of clauses:
\[
\{~\neg p\vee \neg q, \quad \neg p\vee q, \quad p\vee \neg q, \quad p\vee q~\}
\]
\noindent Consider the binary resolution  inference system
$\textrm{BR}$.
Show that there exists an infinite number of different
$\textrm{BR}$ derivations of the empty clause from the clauses of $S$.\bigskip

\Solution{
\noindent {\bf Solution.}

First consider the following derivation of the empty clause.

%\begin{tabular}{rcl}
%1) & $\lnot p \lor \lnot q $\\
%2) & $\lnot p \lor       q $\\
%3) & $      p \lor \lnot q $\\
%4) & $      p \lor       q $\\
%\hline
%5) & $\lnot q \lor \lnot q $& {Binary resolution (1, 3)}\\
%6) & $      q \lor       q $& {Binary resolution (2, 4)}\\
%7) & $      q              $& {Factoring (6)}\\
%8) & $\lnot q              $& {Binary Resolution (6,7)}\\
%9) & $\Box                 $& {Binary Resolution (6,8)}\\
%\end{tabular}

\begin{prooftree}
                \AxiomC{$\lnot p \lor \lnot q$}
                \AxiomC{$      p \lor \lnot q$}
            \BinaryInfC{$\lnot q \lor \lnot q$}
        
                    \AxiomC{$\lnot p \lor       q$}
                    \AxiomC{$      p \lor       q$}
                \BinaryInfC{$      q \lor       q$}
            \UnaryInfC{$      q $}
        \BinaryInfC{$\lnot q$}
                \AxiomC{$\lnot p \lor       q$}
                \AxiomC{$      p \lor       q$}
            \BinaryInfC{$      q \lor       q$}
        \UnaryInfC{$      q $}
        \RightLabel{(*)}
        \BinaryInfC{$\Box$}
\end{prooftree}

\medskip
Then consider the following other derivation:

\begin{prooftree}
                \AxiomC{$\ldots$}
            \UnaryInfC{$q$}
            \AxiomC{$\lnot p \lor \lnot q$}
        \BinaryInfC{$\lnot p$}
        \AxiomC{$p \lor q$}
    \BinaryInfC{$q$}
    \UnaryInfC{$\ldots$}
\end{prooftree}


%\begin{tabular}{rcl}
%I1) & $      q              $& Assumption\\
%\hline
%I2) & $\lnot p              $& {Binary resolution (I1,1)}\\
%I2) & $      q              $& {Binary resolution (I2,4)}\\
%\end{tabular}

%The we can insert the latter derivation arbitrarily often after (7) in the original derivation, hence there is an abritrary number of derications of the empty clause.
The we can insert the latter derivation arbitrarily often at step (*) in the original derivation, hence there is an abritrary number of derivations of the empty clause.


}


\bigskip
\problem{2.2} Consider a well-founded strict ordering $\succ$ on
atoms. Prove that the induced ordering on literals, as defined in
the lecture, is  also well-founded. 


\bigskip



\Solution{
  \noindent{\bf Solution.}
Given a literal $L$, let $atom(L):=p$ if $L$ is $p$ for some atom
$p$. Oherwise, that is if $L$
is $\neg p$, then  $atom(L):=p$.

We extend the ordering $\succ$ on atoms to an ordering to literals, as
defined in the lecture. We denote the such resulting induced ordering
on literals also by $\succ$.

Given two literals $L_i, L_j$, we first show that $L_i\succ L_j$
implies $atom(L_i)\succ atom(L_j)$. We do it by case distinction on
the syntax of $L_i, L_j$. Assume $L_i\succ L_j$ and: 
\begin{itemize}
  \item $L_i$ is $p_i$ and $L_j$ is $p_j$.  In this case,
    $atom(L_i)\succ atom(L_j)$ trivially follows from $L_i\succ L_j$.
    %
  \item $L_i$ is $p_i$ and $L_j$ is $\neg p_j$. As  $L_i\succ L_j$,
    we have $p_i   \succ \neg p_j$. By the extension of $\succ$ over
    literals, we have  $\neg p_i \succ p_i   \succ \neg p_j \succ
    p_j$. As $p_i\succ p_j$, we have $atom(L_i)\succ atom(L_j)$. 
    %
    \item  $L_i$ is $\neg p_i$ and $L_j$ is $ p_j$. This case is
      similar to the previous case.
      
       \item  $L_i$ is $\neg p_i$ and $L_j$ is $\neg p_j$. By the extension of $\succ$ over
    literals, we have $\neg p_i \succ p_i   \succ \neg p_j \succ
    p_j$.   As $p_i\succ p_j$, we have $atom(L_i)\succ atom(L_j)$. 
  \end{itemize}

  We thus conclude that $L_i\succ L_j$ implies $atom(L_i)\succ
  atom(L_j)$.

  Assume now that $\succ$ over literals is not well-founded. This
  means there is an infinite chain of literals:
  \[L_0 \succ L_1 \succ  L_2\succ L_3 \succ \ldots\]
  Then, we have:
   \[atom(L_0) \succ atom(L_1) \succ  atom(L_2)\succ atom(L_3) \succ
     \ldots, \]
   which implies that the ordering $\succ$ over atoms is not
   well-founded. This yields however a contradiction. Hence, $\succ$
   over literals is well-founded too. 
  
  }





\bigskip

%Ordering, selection function, inference
\problem{2.3}
Let $p,q$ be boolean atoms and let $S$ be the following set of ground
formulas:
\[\{~ \neg p \lor \neg q, \quad \neg p \lor q, \quad p\lor \neg q,
  \quad p\lor q~\} \]

Take any ordering such that $p\succ  q$ and any selection function 
$\sigma$ over $S$ such that 
\[\{~\neg p \lor \underline{\neg q}, \quad \underline{\neg p} \lor q, \quad p\lor \underline{\neg q},
  \quad \underline{p}\lor q ~\}.\]


\begin{itemize}
\item[(a)] Is $\sigma$ a well-behaved selection function over $S$? Justify
  your answer!
  % yes. in the first three clauses a negative literal is selected, in the last the maximal literal is selected.
\item[(b)] How many inferences of $\BRis_{\sigma}$ are applicable to 
  $S$? Justify your answer!
  % There are no factoring inference possible, as there are no equal positve literals in any of the clauses.
  % Binary resolution can only be performed between a two selected literals with the same atom and opposite polarity. 
  % The only such literals are in the second and the last clause, hence the only possible inference is ~p \/ q, p \/ q |- q \/ q
\end{itemize}

\medskip
\Solution{
  \noindent{\bf Solution.}

\begin{itemize}
\item[(a)]
    Recall that a well-behaved selection function either selects a
    negative literal, or, if no negative literal is selected, (only)
    all maximal literals are selected in a clause.

    In the first three clauses, a negative literal is selected.
    In the last clause, $p$ is selected which is the (only) maximal literal due to $p \succ q$.
    Hence $\sigma$ is well-behaved.
\item[(b)]
    No factoring inference is possible, because no positive literal appears more than once in any of the clauses.
    Binary resolution can only be performed on clauses where the resolved literal is selected.
    As such, there is only one possible inference (between the second
    and the fourth clause):

    \[
        \prftree{
            \underline{\neg p} \lor q
        }{
            \underline{p}\lor q
        }{
            q \lor q
        }
    \]
\end{itemize}

\medskip
}



\end{document}
