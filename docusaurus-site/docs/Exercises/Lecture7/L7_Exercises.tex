\documentclass[11pt]{article}
\usepackage{exercise}

\usepackage{amsmath,times,color,url}
\usepackage{amsfonts,bussproofs}

\usepackage{alltt}
\usepackage{proof}
\usepackage{tikz}
\usepackage{cancel}
\usepackage{mathtools}



\def\set#1{\{#1\}}
\def\Set#1{\{#1\}}
\def\z{{\tt 0}}
\def\o{{\tt 1}}
\def\t{{\tt 2}}
\def\a{{\tt a}}
\def\b{{\tt b}}
\def\c{{\tt c}}
\def\d{{\tt d}}
\def\DFA{{\sf DFA}}
\def\NFA{{\sf NFA}}
\def\R{{\cal R}}
\def\Picture#1{\strut\centerline{\input #1.eepic}}


\definecolor{midnightblue}{cmyk}{0.98,0.13,0,0.43}
\definecolor{olivegreen}{cmyk}{0.64,0,0.95,0.40}
\definecolor{rawsienna}{cmyk}{0,0.72,1,0.45}
\definecolor{fuchsia}{cmyk}{0.47,0.91,0,0.08}
\definecolor{shaded}{rgb}{0.7, 0.75, 0.71}


\newcommand{\Blue}[1]{{\color{blue}#1}}
\newcommand{\Red}[1]{{\color{red}#1}}
\newcommand{\Brown}[1]{{\color{brown}#1}}
\newcommand{\OliveGreen}[1]{{\color{olivegreen}#1}}
\newcommand{\RawSienna}[1]{{\color{rawsienna}#1}}
\newcommand{\MidnightBlue}[1]{{\color{midnightblue}#1}}
\newcommand{\Fuchsia}[1]{{\color{fuchsia}#1}}
\newcommand{\Gray}[1]{{\color{shaded}#1}}

\usepackage{prftree}


\newcommand{\M}[1]{{\color{midnightblue}#1}}
\newcommand{\notl}{\neg}
\newcommand{\orl}{\vee}
\newcommand{\andl}{\wedge}
\newcommand\impl{\rightarrow}

\newcommand{\setof}[1]{\{#1\}}
\newcommand{\true}{1}
\newcommand{\false}{0}

\newcommand{\Th}{\mathcal{T}}

\newcommand{\BRr}{\mathrm{B}\mathrm{R}}   
\newcommand{\Sup}{\mathrm{S}\mathrm{up}}   
\newcommand{\SUPiss}{\mathrm{S}\mathrm{up}_{\succ,\sel}}   % SUPis with selection
\newcommand{\sel}{\sigma}


% object level equality
\newcommand{\oeq}{=}%\approx}
\newcommand{\oneq}{\neq}%\not\approx}

\newcommand{\SUPis}{\mathbb{S}\mathrm{up}}   % SUP
\newcommand{\Iis}{\mathbb{I}} %inference system


\newcommand{\BRis}{\mathbb{BR}}             % binary resolution inference system
%\newcommand{\Solution}[1]{#1}
\newcommand{\Solution}[1]{}

\newcommand{\Union}{\bigcup}


\begin{document}
\problemset{Lecture 7, November 3, 2025}


\section*{\exercises{7}}

\problem{7.1}
Consider the following set $S$ of clauses:
\[
\begin{array}{l}
\neg p(z,a) \orl \neg p(z,x) \orl \neg p(x,z)\\
p(y,a) \orl p(y,f(y))\\
p(w,a) \orl p(f(w),w)
\end{array}
\]
where $p$ is a predicate symbol, $f$ is a function symbol, $x,y,z,w$ are variables and $a$ is a
constant. 

\noindent Give a refutation proof of $S$ by using the non-ground binary resolution
inference system $\BRis$. For each newly derived clause, 
label the clauses from which it was derived by which inference
rule  and indicate
most general unifiers.

\Solution{
  \bigskip

\noindent
{\bf Solution:}

For simplicity, we name the given clauses by numbers: 
\[
\begin{array}{ll}
(1) & \neg p(z,a) \orl \neg p(z,x) \orl \neg p(x,z)\\
(2) & p(y,a) \orl p(y,f(y))\\
(3)& p(w,a) \orl p(f(w),w)
\end{array}
\]


By negative factoring on (1), with the mgu $\{x\to a\}$, we get: 
\[(4) \quad \neg p(z,a) \orl \neg p(a,z)\]

By negative factoring on (4), with the mgu $\{z\to a\}$, we get: 
\[(5) \quad \neg p(a,a)\]

By resolution on (5) and (2), with the mgu $\{y\to a\}$, we get:
\[(6) \quad  p(a,f(a))\]

By resolution on (4) and (6), with the mgu $\{z\to f(a)\}$, we get:
\[(7) \quad \neg p(f(a), a)\]

By resolution on (3) and (7), with the mgu $\{w\to a\}$, we get:
\[(8) \quad  p(a, a)\]

By resolution on (5) and (8), we finally obtain the empty clause:
\[(9) \quad  \Box\]

Hence, our input set $S$ of clauses (1), (2) and (3) is
unsatisfiable.
}

\bigskip



\problem{7.2}
Let $p$ denote  a unary predicate symbol, $f$  a unary function
symbol, $x,y$ variables and $c$ a constant.
Let $C_1$ be the clause $p(x) \vee p(y)$ and consider $C_2$ to be the
clause $p(x)$. Further, let $D$ denote the clause $p(f(c))$. % We
%consider $\succ$ to be an ordering over literals; for simplicity, we
%write $\succ$ also  for the bag extension of $\succ$ for comparing
%clauses. 


\begin{itemize}
\item[(a)] Does $C_1$ subsume $D$?
\item[(b)] Does $C_2$ subsume $D$?
\end{itemize}

Justify your answers!
\medskip

\Solution{
{\bf Solution:}
\begin{itemize}
  \item[(a)] No. For $C_1$ to subsume $D$, there must be a substitution $\theta$ such
    that $C_1\theta$ is a sub-multiset of $D$. However, since $C_1$ has two literals, also $C_1\theta$
    has two literals and therefore it cannot be a sub-multiset of $D$, which only has one literal.
  %  (Note that for a substitution $\theta = \{x\mapsto f(c), y\mapsto f(c)\}$, it holds that $C\theta\succ D$.
   % Therefore, while $C\theta$ is a \emph{subset} of $D$, it is not a \emph{sub-multiset}.) 
  \item[(b)] Yes. For $\theta = \{x\mapsto f(c)\}$, it holds that $C_2\theta$ is a sub-multiset of $D$.
\end{itemize}

}



%non-ground sup
\problem{7.3}
Let $x$ denote a variable, $a,b,c$ constants, and $f$  a
 unary function symbol. \\
Give a superposition refutation of the following set of two clauses:

\[
  \begin{array}{lll}
    \{~ &x \oeq f(c),&\\
    &a\oneq b &~\}
   \end{array}
 \]
 such that, in every inference, the premises and
 the conclusion of that inference do not use the symbols $f,c$
 together with the symbols $a,b$. That is, 
 every inference has the following property: if the premise or the
 conclusion contain any of the symbols $f,c$,
 then the premise and the conclusion contain neither $a$ nor $b$.
 
In your proof, use only the inference system of the superposition
calculus $\SUPis$ (without ordering and selection function); that is,
no inferences of binary resolution $\BRis$ should be used.
For each newly derived clause,
clearly label the clauses from which it was derived and indicate most general unifiers.

\medskip

\Solution
{\noindent{\bf Solution.}
Note that if we apply a rule to two clauses, we consider the variables in both clauses to be distinct, even if the two clauses are actually the same one. 
Therefore we when we apply can use superposition using $f(c) = x$, as both of the premises it is the same thing as applying it to the premises $f(c) = x$ and $f(c) = y$.
This means we can do the following derivation:

\medskip

$
  \begin{array}{lcll}
    1) & f(c) = x & \text{Axiom}\\
    2) & a \neq b & \text{Axiom}\\
    3) & x = y & \text{Superposition using (1), and (1), $\sigma = \emptyset$}\\
    4) & a \neq x & \text{Superposition using (2), and (3), $\sigma = \Set{ y \mapsto b }$}\\
    5) & \Box & \text{Equality resolution using (4), $\sigma = \Set{ x \mapsto a }$}\\
  \end{array}
$
}
\medskip











\end{document}