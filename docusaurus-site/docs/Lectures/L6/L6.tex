
\documentclass[10pt]{beamer}
\usepackage{times}
\usepackage{alltt}
\usepackage{proof}
\usepackage{tikz}

\input{dvipsnam.def}
\newenvironment{VampireProof}{%
	\section{Proof}}{}
\newenvironment{VampireInference}{%
	\begin{array}{c}}{\end{array}}
\newenvironment{VampireInferencePremises}{}{}
\newenvironment{VampirePremise}%
	{\begin{array}{l}}%
	{\end{array}}
\newenvironment{VampireConclusion}%
	{\begin{array}{l}}%
	{\end{array}}
\newcommand{\VampireUnit}[3]{%
	#1.~#2~[#3]}

\newcommand{\VPremiseSeparator}{\\}
%\newcommand{\VConclusionSeparator}{\\ \longrightarrow \\}
\newcommand{\VConclusionSeparator}{\\ \hline}

\newcommand{\Vor}{\vee}
\newcommand{\Vand}{\wedge}
\newcommand{\Vimp}{\supset}
\newcommand{\Viff}{\equiv}
\newcommand{\Vxor}{\not\equiv}

\newcommand{\VEmptyClause}{\Box}


\newcommand{\sLide}[1]{\begin{frame}\frametitle{#1}}

\AtBeginSection[]{\frame{\frametitle{Outline}\tableofcontents[current]}}

\setbeamercovered{dynamic}
\setbeamercolor{math text}{fg=MidnightBlue}

\definecolor{midnightblue}{cmyk}{0.98,0.13,0,0.43}
%\definecolor{olivegreen}{cmyk}{0.64,0,0.95,0.40}
\definecolor{rawsienna}{cmyk}{0,0.72,1,0.45}
\definecolor{fuchsia}{cmyk}{0.47,0.91,0,0.08}
\colorlet{redshaded}{red!25!bg}
\colorlet{shaded}{black!25!bg}
\colorlet{shadedshaded}{black!10!bg}
\colorlet{blackshaded}{black!40!bg}

\newcommand{\OliveGreen}[1]{{\color{OliveGreen}#1}}
\newcommand{\RedViolet}[1]{{\color{RedViolet}#1}}
\newcommand{\Mulberry}[1]{{\color{Mulberry}#1}}
\newcommand{\Fuchsia}[1]{{\color{Fuchsia}#1}}
\newcommand{\Brown}[1]{{\color{Brown}#1}}
\newcommand{\Black}[1]{{\color{Black}#1}}
\newcommand{\Magenta}[1]{{\color{Magenta}#1}}
\newcommand{\Green}[1]{{\color{Green}#1}}
\newcommand{\Purple}[1]{{\color{Purple}#1}}
\newcommand{\OrangeRed}[1]{{\color{OrangeRed}#1}}
\newcommand{\RoyalBlue}[1]{{\color{RoyalBlue}#1}}
\newcommand{\MidnightBlue}[1]{{\color{MidnightBlue}#1}}
\newcommand{\DarkOrchid}[1]{{\color{DarkOrchid}#1}}
\newcommand{\RedOrange}[1]{{\color{RedOrange}#1}}
\newcommand{\PineGreen}[1]{{\color{PineGreen}#1}}
\newcommand{\WildStrawberry}[1]{{\color{WildStrawberry}#1}}
\newcommand{\RawSienna}[1]{{\color{RawSienna}#1}}
\newcommand{\Periwinkle}[1]{{\color{Periwinkle}#1}}

\newcommand{\xone}[2]{{#1_{1},\ldots,#1_{#2}}}
\newcommand{\setof}[1]{{\{#1\}}}
\newcommand{\tuple}[1]{{\langle#1\rangle}}
\newcommand{\M}{}
\newcommand{\vs}{\visible}
\newcommand{\DI}[2]{\alert{#1}}
\newcommand{\DII}[3]{\alert{#1}}
\newcommand{\RR}{\Rightarrow}

%
%
% Algorithms
%
%

\newcommand{\inc}{~~~\= \+ \kill}    % used in algorithms
\newcommand{\dec}{\- \kill}         % used in algorithms

\newcommand{\reserved}[1]{\OliveGreen{\texttt{\bf #1}}}
\newcommand{\semicol}{\textbf{;}}                  % semicolon in algorithms
\newcommand{\commentinalg}[1]{\texttt{#1}}         % comment in algorithms
\newcommand{\PROCEDURE}{\reserved{procedure}}
\newcommand{\SUBPROCEDURE}{\reserved{subprocedure}}
\newcommand{\PARAMETERS}{\reserved{parameters}}
\newcommand{\INPUT}{\reserved{input}}
\newcommand{\LOOP}{\reserved{loop}}
\newcommand{\OUTPUT}{\reserved{output}}
\newcommand{\IF}{\reserved{if}}
\newcommand{\CASE}{\reserved{case}}
\newcommand{\OF}{\reserved{of}}
\newcommand{\DO}{\reserved{do}}
\newcommand{\OD}{\reserved{od}}
\newcommand{\THEN}{\reserved{then}}
\newcommand{\ELSE}{\reserved{else}}
\newcommand{\WHILE}{\reserved{while}}
\newcommand{\BEGIN}{\reserved{begin}}
\newcommand{\END}{\reserved{end}}
\newcommand{\LET}{\reserved{let}}
\newcommand{\FORALL}{\reserved{forall}}
\newcommand{\ASS}{\texttt{ := }}
\newcommand{\RETURN}{\reserved{return}}
\newcommand{\REPEAT}{\reserved{repeat}}
\newcommand{\VAR}{\reserved{var}}

\newcommand{\vampNot}{\~{}}    % negation in Vampire's output
\newcommand{\Active}{\mathit{active}}
\newcommand{\Given}{\mathit{given}}
\newcommand{\Unprocessed}{\mathit{unprocessed}}
\newcommand{\Passive}{\mathit{passive}}
\newcommand{\Current}{\mathit{current}}
\newcommand{\Init}{\mathit{init}}
\newcommand{\New}{\mathit{new}}
\newcommand{\Select}{\mathit{select}}
\newcommand{\Infer}{\mathit{infer}}
\newcommand{\Queue}{\mathit{queue}}
\newcommand{\Pop}{\mathit{pop}}
%\newcommand{\Mark}{\ding{52}}
\newcommand{\Mark}{\texttt{*}}
\newcommand{\Simplify}{\mathit{simplify}}
\newcommand{\Retained}{\mathit{retained}}
\newcommand{\ForwardSimplify}{\mathit{forward\_simplify}}
\newcommand{\Process}{\mathit{process}}
\newcommand{\BackwardSimplify}{\mathit{backward\_simplify}}
\newcommand{\GoalFound}{\mathit{goal\_found}}
\newcommand{\opt}[1]{\texttt{#1}}
\newcommand{\option}[1]{\opt{-#1}}
\newcommand{\Option}[1]{\opt{--#1}}
\newcommand{\optionval}[2]{\texttt{-#1~#2}}
\newcommand{\Optionval}[2]{\texttt{--#1~#2}}
\newcommand{\optionI}[1]{\texttt{-#1}\index{#1@\texttt{-#1}}}
\newcommand{\OptionI}[1]{\texttt{--#1}\index{#1@\texttt{--#1}}}
\newcommand{\optionvalI}[2]{\texttt{-#1~#2}\index{#1@\texttt{-#1}}}
\newcommand{\OptionvalI}[2]{\texttt{--#1~#2}\index{#1@\texttt{--#1}}}
\newcommand{\assign}{\texttt{:=}}                  % assignment in algorithms

%
% Multisets
%

\newcommand{\msminus}{\mathbin{\protect\dot{-}}} % multiset minus
\newcommand{\msin}{\mathbin{\protect\dot{\in}}}  % multiset membership
\newcommand{\msplus}{\mathbin{\protect\dot{+}}}  % multiset plues
\newcommand{\mscup}{\mathbin{\protect\dot{\cup}}}% multiset union
\newcommand{\mstimes}{\mathbin{\protect\dot{\times}}}% multiset product
\newcommand{\mscap}{\mathbin{\protect\dot{\cap}}}% multiset intersection
\newcommand{\mssubseteq}{\mathbin{\protect\dot{\subseteq}}} % sub multiset
\newcommand{\mssetof}[1]{{\protect\dot{\{}#1\protect\dot{\}}}} % multiset (explicit listing)
\newcommand{\mssetofb}{\protect\dot{\{}}     % multiset begin
\newcommand{\mssetofe}{\protect\dot{\}}}     % multiset end
\newcommand{\sizeof}[1]{|#1|}                % size of a multiset
                                % multiset-to-set conversion
\newcommand{\MtoS}[1]{\protect\stackrel{\curlywedge}{#1}}
                                % set-to-multiset conversion
\newcommand{\StoM}[1]{\protect\stackrel{\curlyvee}{#1}}

\newcommand{\eqbydef}{\stackrel{\text{def}}{=}}
\newcommand{\bydef}{\stackrel{\text{def}}{\Leftrightarrow}}

\newcommand{\rr}{\rightarrow}               % rewrite
\newcommand{\irr}{\leftarrow}               % inverse of rewrite
\newcommand{\nfrr}{\rr^!}                 % normal form rewrites
\newcommand{\nf}[1]{#1\downarrow}         % normal form binary relation

%
%
%   Logical notation
%
%
\newcommand{\imply}{\rightarrow}
\renewcommand{\implies}{\rightarrow}
\newcommand{\orl}{\vee}
\newcommand{\bigorl}{\bigvee}
\newcommand{\andl}{\wedge}
\newcommand{\bigandl}{\bigwedge}
\newcommand{\notl}{\neg}
\newcommand{\iffl}{\leftrightarrow}
\newcommand{\prov}{\vdash}
\newcommand{\Subst}[1]{\{#1\}}   % for substitutions
\newcommand{\subst}[2]{#1 \mapsto #2}   % for substitutions
\newcommand{\mgu}{\mathit{mgu}}
\newcommand{\mgs}{\mathit{mgs}}
\renewenvironment{proof}{
    {\em Proof.\/}}{
    \hfill $\Box${\vspace{4mm}}}
\newcommand{\true}{\mathrm{true}}
\newcommand{\false}{\mathrm{false}}
\newcommand{\intI}{I}                  % interpretation
\newcommand{\emptyclause}{\Box}

\newcommand{\ruleSup}{\mathsf{Sup}}
\newcommand{\ruleDem}{\mathsf{Dem}}
\newcommand{\ruleER}{\mathsf{ER}}
\newcommand{\ruleEF}{\mathsf{EF}}
\newcommand{\ruleFact}{\mathsf{Fact}}
\newcommand{\eql}{\simeq}
\newcommand{\neql}{\neq}
\newcommand{\deleted}[1]{#1}

%
%    Resolution
%

\newcommand{\mextension}[1]{#1^{\mathit{bag}}} % multiset extension of #1
\newcommand{\succm}{\mextension{\succ}}        % multiset extension of \succ
\newcommand{\succlm}{\mextension{\succl}}      % multiset extension of \succl
\newcommand{\succl}{\succ_{\mathit{lit}}}     % literal extension of \succ
\newcommand{\succeql}{\succeq_{\mathit{lit}}} % literal extension of \succ
\newcommand{\sel}{\sigma}                   % selection function
\newcommand{\BRr}{\mathsf{BR}}              % binary resolution rule
\newcommand{\Fact}{\mathsf{Fact}}           % factoring rule
\newcommand{\BRis}{\mathbb{BR}}             % binary resolution inference system
\newcommand{\SUPis}{\mathbb{S}\mathrm{up}}           % superposition inference system
\newcommand{\isI}{\mathbb{I}}               % inference system I
\newcommand{\sis}[1]{#1^{\mathrm{sat}}}     % saturating inference system bazed on I
\newcommand{\BRiss}{\mathbb{BR}_{\sel}}     % binary resolution inference system with selection
\newcommand{\SUPiss}{\mathbb{S}\mathrm{up}_{\succ,\sel}}   % SUPis with selection
\newcommand{\TA}{\mathit{TA}}               % term algebra
\newcommand{\nat}{\mathbb{N}}               % natural numbers
\newcommand{\KBo}{\succ_{\mathit{KB}}}   % Knuth-Bendix ordering
\newcommand{\wf}{w}               % weight function
\newcommand{\weight}[1]{|#1|}     % weight of term


\colorlet{darkred}{red!80!black}
\colorlet{darkblue}{blue!80!black}
\colorlet{darkgreen}{green!80!black}
\colorlet{darkergreen}{green!60!black}

\def\radius{0.96cm}
\def\innerradius{0.85cm}

\def\softness{0.4}
\definecolor{softred}{rgb}{1,\softness,\softness}
\definecolor{softgreen}{rgb}{\softness,1,\softness}
\definecolor{softblue}{rgb}{\softness,\softness,1}

\definecolor{softrg}{rgb}{1,1,\softness}
\definecolor{softrb}{rgb}{1,\softness,1}
\definecolor{softgb}{rgb}{\softness,1,1}

\author{
  \textcolor{red!90!black}{Laura Kov\'acs}}
\titlegraphic{\hspace*{1em}\includegraphics[width=.5\textwidth]{forsyte}}

\newcommand{\Blue}[1]{{\color{blue}#1}}

%
% Misc
%
\newcommand{\clr}[2]{\mathit{\OliveGreen{#1}\Red{#2}}}

\title{
              	       Automated Deduction
}

%
%
\author{
  \textcolor{red!90!black}{Laura Kov\'acs} \\
\titlegraphic{\hspace*{1em}\includegraphics{forsyte}}}
\begin{document}

\date{}
\frame{\titlepage}





\begin{frame}

\frametitle{First-Order Logic: Exercises}

Which of the following statements are true?

\begin{enumerate}
\item\visible<2->{First-order logic is an \alert{extension of propositional logic};}

\item\visible<3->{First-order logic is \alert{NP-complete.}}

\item\visible<4->{In first-order logic you can use \alert{quantifiers over sets.}}

\item\visible<5->{First-order logic is \alert{decidable}.}

\item\visible<6-> {One can \alert{axiomatise naturals} in first-order logic;}

\item\visible<7-> {Having proofs is  \alert{good}}

\end{enumerate}


\end{frame}
%------------------------------------------------------------



\section{Unification and Lifting}

%---------------------------------------------------------------------

                        \begin{frame}\frametitle{Substitution}

\begin{itemize}
\item
A \alert{substitution} $\M{\theta}$ is a mapping from variables to terms
such that the set $\M{\setof{x \mid \theta(x) \neq x}}$ is finite. 

\item
This set is called the \alert{domain} of $\M{\theta}$.

\item Notation: 
$\alert{\Subst{\subst{x_1}{t_1},\ldots,\subst{x_n}{t_n}}}$,
where $\M{\xone{x}{n}}$ are pairwise different variables,
denotes the substitution $\M{\theta}$ such that

  \[\M{
    \theta(x) = \left\{
                  \begin{array}{ll}
	            t_i & \text{if } x = x_i; \\
                    x   & \text{if } x \not\in \setof{\xone{x}{n}}.
                  \end{array}
                \right.
  }\]

\item
\alert{Application of this substitution to an expression} $\M E$: 
simultaneous replacement of $\M{x_i}$ by $\M{t_i}$.

\item<2->
Application of a substitution $\M{\theta}$ to $\M E$ is denoted by
$\alert{E\theta}$.

\item<2-> Since substitutions are functions, we can define their
\alert{composition} (written $\alert{\sigma\tau}$ instead of $\M{\tau
\circ \sigma}$). Note that we have $\M{E(\sigma\tau) = (E\sigma)\tau}$.
\end{itemize}


                                \end{frame}

%---------------------------------------------------------------------
                           \begin{frame}\frametitle{Example}
Consider: 
\[
\begin{array}{l}
 E=p(x,y,f(a))\\
 \theta=\Subst{\subst{x}{b},\subst{y}{x}}
\end{array}\]

What is $E\theta$?
                             \end{frame}


%---------------------------------------------------------------------
                           \begin{frame}\frametitle{Substitution composition}


Suppose we have two substitutions

  \[\M{
    \begin{array}{l}
      \theta_1=\Subst{\subst{x_1}{s_1},\ldots,\subst{x_m}{s_m}} \text{ and } \\
      \theta_2=\Subst{\subst{y_1}{t_1},\ldots,\subst{y_n}{t_n}}.
    \end{array}
  }\]
How can we compute their composition $\M{\theta_1 \theta_2}$?

\bigskip


\vs<2->{
The substitution $\M{\theta_1\theta_2}$ is obtained from the set: 

\[\M{
    \begin{array}{l}
      \Subst{\subst{x_1}{s_1\theta_2},\ldots,\subst{x_m}{s_m\theta_2},\\
      ~~\subst{y_1}{t_1},\ldots,\subst{y_n}{t_n}},
    \end{array}
  }\]
\vs<3->{by deleting
\begin{itemize}
\item all $\subst{y_i}{t_i}$ with $y_i\in\{x_1, \ldots,x_m\}$,
\item all $\subst{x_i}{s_i\theta_2}$ with $x_i=s_i\theta_2$.
\end{itemize}}

}

                                \end{frame}

%---------------------------------------------------------------------
                           \begin{frame}\frametitle{Example}

Consider: 

\[
\begin{array}{l}
\theta_1=\Subst{\subst{x}{f(y)}, \subst{y}{z}}, \\
\theta_2=\Subst{\subst{x}{a}, \subst{y}{b}, \subst{z}{y}}.
\end{array}
\]

\bigskip

What is $\theta_1\theta_2$?
  \end{frame}

%---------------------------------------------------------------------
  

                       \begin{frame}\frametitle{Instances, Ground}

An \alert{instance} of an expression (that is term, atom, literal, or
clause) $\M E$
is obtained by applying a substitution to $\M E$. Examples:

\begin{itemize}
\item some instances of the term $\M{f(\Magenta{x},a,g(\Magenta{x}))}$ are: \\
  $\M{f(\Magenta{x},a,g(\Magenta{x}))}$, \\
  $\M{f(\Magenta{y},a,g(\Magenta{y}))}$, \\
  $\M{f(\Magenta{a},a,g(\Magenta{a}))}$, \\
  $\M{f(\Magenta{g(b)},a,g(\Magenta{g(b)}))}$;
\item but the term 
  $\M{f(\Magenta{b},a,g(\Magenta{c}))}$ is not an instance of this term.
\end{itemize}
\alert{Ground instance:} instance with no variables.


                                \end{frame}



%---------------------------------------------------------------------

                        \begin{frame}\frametitle{Herbrand's Theorem}

For a set of clauses $\M S$ denote by \alert{$S^*$} the set of ground
instances of clauses in $\M S$.

\bigskip

\textbf{\OliveGreen{Theorem}}
Let $\Sigma$ be a signature with at least one constant symbol and $\M
S$ be a set of (universal) clauses over $\Sigma$.
The following conditions are equivalent.
\begin{enumerate}
\item \Blue{$\M S$ is unsatisfiable};
\item \Blue{$\M{S^*}$ is unsatisfiable};
\end{enumerate}

\vs<2->{
By compactness of first-order logic the last condition is equivalent to

\begin{enumerate}\setcounter{enumi}{2}%
\item \Blue{there exists a finite unsatisfiable set of
ground instances of clauses in $\M S$}.
\end{enumerate}
}
\bigskip

\vs<3->{
The theorem reduces the problem of checking unsatisfiability of sets of 
\Magenta{arbitrary} clauses to checking
unsatisfiability of sets of \alert{ground} clauses \ldots

\medskip
\vs<4->{The only
problem is that $\M{S^*}$ can be infinite even if
$\M{S}$ is finite.}
}
                                \end{frame}

%---------------------------------------------------------------------

%       \begin{frame}\frametitle{Note on Herbrand's Theorem, Compactness 
%         and Completeness}

%The proofs of completeness of resolution and superposition with
%redundancy elimination \alert{does
%  not use any of these theorems.} 

%\bigskip

%\vs<2->{
%  Interestingly, \Blue{they all can be derived as simple corollaries} of this
%  proof of completeness!
%}

%                                \end{frame}

%---------------------------------------------------------------------

                              \begin{frame}\frametitle{Lifting}

\alert{Lifting} is a technique for proving completeness theorems
in the following way:

\begin{enumerate}
\item Prove completeness of the system for a set of \Blue{ground}
clauses;
\item \alert{Lift} the proof to the non-ground case.
\end{enumerate}

                                \end{frame}

%---------------------------------------------------------------------


                       	   \begin{frame}\frametitle{Lifting, Example}

Consider two (non-ground) clauses $\M{p(x,a) \orl q_1(x)}$ and
$\M{\notl p(y,z) \orl q_2(y,z)}$. If the signature
contains function symbols, then both clauses have infinite sets of
instances:

\[\M{
  \begin{array}{rcl}
    \{p(r,a) \orl q_1(r) & \mid & r \text{ is ground}\} \\
    \{\notl p(s,t) \orl q_2(s,t) & \mid & s,t \text{ are ground}\}
  \end{array}}
\]

We can resolve such instances if and only if $\M{r = s}$ and
$\M{t=a}$. Then we can apply the following inference

\[\M{
  \infer[(\BRr)]{q_1(s) \orl q_2(s,a)}{
               p(s,a) \orl q_1(s) &
               \notl p(s,a) \orl q_2(s,a)}
}\]

But \Blue{there is an infinite number of such inferences.}


                                \end{frame}

%---------------------------------------------------------------------

                       	   \begin{frame}\frametitle{Lifting, Idea}

The idea is to represent an \alert{infinite number of ground inferences} of
the form

\[\M{
  \infer[(\BRr)]{q_1(s) \orl q_2(s,a)}{
               p(s,a) \orl q_1(s) &
               \notl p(s,a) \orl q_2(s,a)}
}\]

by a \alert{single non-ground inference}

\[\M{
  \infer[(\BRr)]{q_1(y) \orl q_2(y,a)}{
               p(x,a) \orl q_1(x) &
               \notl p(y,z) \orl q_2(y,z)}
}\]

\OliveGreen{Is this always possible?}


                           \end{frame}
%---------------------------------------------------------------------

                       	   \begin{frame}\frametitle{Yes!}


\[\M{
  \infer[(\BRr)]{q_1(y) \orl q_2(y,a)}{
               p(x,a) \orl q_1(x) &
               \notl p(y,z) \orl q_2(y,z)}
}\]


Note that the substitution $\M{\setof{x \mapsto y,z\mapsto a}}$
is a solution of the ``equation'' $\M{p(x,a) = p(y,z)}$.


                           \end{frame}

%---------------------------------------------------------------------

%---------------------------------------------------------------------

      \begin{frame}\frametitle{Lifting
                               \vs<2->{(Robinson,
                                 1965)\vs<4->{(Bachmair \& Ganzinger, 1990)}}}

                             \begin{overprint}
  \onslide<1-2>
\begin{block}{}
  \Blue{Idea:} Represent an \alert{infinite number of ground
  inferences}
by a \alert{single non-ground inference}.

\vspace*{1em}

 \vs<2->{ In case of $\BRis$:
\begin{itemize}
  \item Resolution for \alert{non-ground} clauses 
  \item The notion of \alert{``same''} ground atoms is generalized to \alert{unifiability} of
    non-ground atoms;
    \item Only compute \alert{substitutions} that are \alert{most general unifiers
      (mgu)}.
  \end{itemize}
  }
  \end{block}
  
   \onslide<3->
    
    \begin{block}{Lifting Lemma for $\BRr$ in $\BRis$:}<3->
      Let $C$ and $D$ clauses with no shared variables. If:

\[\M{
  \infer[(ground~\BRr)]{C'}{\begin{array}{l}C \\[.5em]\big\downarrow
                                     \sigma_1\\[.5em]
                                     C\sigma_1\end{array}& \begin{array}{l}D
                                                             \\[.5em]\big\downarrow
                                                             \sigma_2\\[.5em] D\sigma_2\end{array}}
}\]

then there exists a substitution $\sigma$ sucht that:

\[\M{
  \infer[(non-ground~\BRr)]{\begin{array}{c}C''
                                    \\[.5em]\big\downarrow
                                    \sigma\\[.5em]
                                    C'=C''\sigma\end{array}}{C  & D}
}\]
      
\end{block}



\vs<4->{\alert{ Similar lifting lemmas each inferences of $\BRis$ and
$\SUPis$.} }
    \end{overprint}
                           \end{frame}


%-------------------------------------------------------------------------                           
              	   \begin{frame}\frametitle{What should we lift?}

\begin{itemize}
\item Ordering $\M{\succ}$;
\item Selection function $\M{\sel}$;
\item Calculus $\M{\SUPiss}$.
\end{itemize}

Most importantly, for the lifting to work we should be able to \alert{solve
equations $\M{s = t}$} between terms and between atoms. This can be done
using \alert{most general unifiers.}

                                \end{frame}
%---------------------------------------------------------------------

                      \begin{frame}\frametitle{Unifier}

\alert{Unifier} of expressions $\M{s_1}$ and $\M{s_2}$: a substitution
$\M{\theta}$ such that $\M{s_1\theta=s_2\theta}$.

In other words, a unifier is a \alert{solution to an ``equation''}
$\M{s_1 = s_2}$. In a similar way we can define solutions to
systems of equations $\M{s_1 = s_1',\ldots,s_n=s_n'}$.
We call such solutions \alert{simultaneous unifiers} of 
$\M{\xone{s}{n}}$ and $\M{\xone{s'}{n}}$.

                                \end{frame}

%---------------------------------------------------------------------

                      \begin{frame}\frametitle{(Most General) Unifiers}

A solution $\M\theta$ to a set of equations $\M{E}$ is said to be
a \alert{most general solution} if for every other solution $\M\sigma$
there exists a substitution $\M\tau$ such that
$\M{\theta\tau = \sigma}$. 
In a similar way can define a \alert{most general unifier}.

\bigskip

\vs<2->{
Consider terms $\M{f(x_1,g(x_1),x_2)}$ and $\M{f(y_1,y_2,y_2)}$.

(Some of) their unifiers are \\
$\Blue{\theta_1 = \Subst{\subst{y_1}{x_1},\subst{y_2}{g(x_1)},\subst{x_2}{g(x_1)}}}$ and \\
$\Fuchsia{\theta_2 = \Subst{\subst{y_1}{a},\subst{y_2}{g(a)},\subst{x_2}{g(a)},\subst{x_1}{a}}}$:

\medskip

$\Blue{f(x_1,g(x_1),x_2)\theta_1 = f(x_1,g(x_1),g(x_1))}$;\\
$\Blue{f(y_1,y_2,y_2)\theta_1 = f(x_1,g(x_1),g(x_1))}$;\\
$\Fuchsia{f(x_1,g(x_1),x_2)\theta_2 = f(a,g(a),g(a))}$;\\
$\Fuchsia{f(y_1,y_2,y_2)\theta_2 = f(a,g(a),g(a))}$.

\smallskip

But only $\RawSienna{\theta_1}$ is \alert{most general}.
}
                                \end{frame}

%---------------------------------------------------------------------


              \begin{frame}\frametitle{Unification}

\footnotesize
Let $\M E$ be a set of equations. An
\alert{isolated equation in $E$} is any equation $\M{x=t}$ in $E$ such
that $\M x$ has exactly one occurrence in $\M E$.\vspace*{-.4em}

\begin{tt}
\begin{tabbing}
\INPUT: \\[-0.4ex] \inc
  \rm A finite set of equations $\M{E}$\\
\rm ($s,t$ denote terms,     $c,d$ constants, $f,g$ function symbols, $x$  variable)\\[-0.4ex] \dec
\OUTPUT: \\[-0.4ex] \inc
  \rm A solution to $\M{E}$ or failure. \\[-0.4ex] \dec
\BEGIN \\[-0.4ex] \inc
  \WHILE\ {\rm there exists a non-isolated  equation $\M{(s = t) \in E}$} \\[-0.4ex]
  \DO\ \\[-0.4ex] \inc
    \CASE\ $\M{(s,t)}$ \OF\ \\[-0.4ex] \inc
      $\M{(t,t)}$ $\Rightarrow$ \rm Remove this equation from $\M{E}$ \\[-0.4ex]
      $\M{(x,t)}$ $\Rightarrow$ \\[-0.4ex] \inc
        \IF\ \rm $\M{x}$ occurs in $\M{t}$ \\[-0.4ex]
        \THEN\ \rm halt with failure \\[-0.4ex]
        \ELSE\ \rm replace $\M{x}$ by $\M{t}$ in all other equations of $\M{E}$ \\[-0.4ex] \dec
      $\M{(t,x)}$ $\Rightarrow$ \= \rm replace this equation by $\M{x = t}$ \+\\[-0.4ex]
                               \rm and do the same as in the case $\M{(x,t)}$
                                                                \\[-0.4ex] \dec
      $\M{(c,d)}$ $\Rightarrow$ \rm halt with failure \\[-0.4ex]
      $\M{(c,f(\xone{t}{n}))}$ $\Rightarrow$ \rm halt with failure \\[-0.4ex]
      $\M{(f(\xone{t}{n}),c)}$ $\Rightarrow$ \rm halt with failure \\[-0.4ex]
      $\M{(f(\xone{s}{m}),g(\xone{t}{n}))}$ $\Rightarrow$ \rm halt
            with failure \\[-0.4ex]
      $\M{(f(\xone{s}{n}),f(\xone{t}{n}))}$ $\Rightarrow$
                \= \rm replace this equation by the set \+\\[-0.4ex]
                   $\M{s_1 = t_1,\ldots,s_n = t_n}$
                \\[-0.4ex] \dec
    \END\ \\[-0.4ex] \dec
  \OD\ \\[-0.4ex]
  \rm Now $\M{E}$ has the form $\M{\setof{x_1 = r_1,\ldots,x_l =
r_l}}$ and every equation in it \\[-0.4ex]
                    \rm is isolated \\[-0.4ex]
   \RETURN\ \rm the substitution
    $\M{\Subst{\subst{x_1}{r_1},\ldots,\subst{x_l}{r_l}}}$ \\[-0.4ex] \dec
\END
\end{tabbing}
\end{tt}


                                \end{frame}

%---------------------------------------------------------------------


              \begin{frame}\frametitle{Examples}

  \[\M{
    \begin{array}{l}
      \setof{h(g(f(x),a)) = h(g(y,y))} \\
      \setof{h(f(y),y,f(z)) = h(z,f(x),x)} \\
      \setof{h(g(f(x),z)) = h(g(y,y))} \\
    \end{array}
  }\]

                                \end{frame}

%---------------------------------------------------------------------


                  \begin{frame}\frametitle{Properties}


\textbf{\OliveGreen{Theorem}}
Suppose we run the unification algorithm on $\M{s = t}$. Then

\begin{itemize}
\item If $\M{s}$ and $\M{t}$ are unifiable, then the algorithms
terminates and outputs a most general unifier of $\M{s}$ and $\M{t}$.

\item If $\M{s}$ and $\M{t}$ are not unifiable, then the algorithms
terminates with failure.
\end{itemize}

Notation (slightly ambiguous):

\begin{itemize}
\item $\alert{\mgu(s,t)}$ for a most general unifier;
\item $\alert{\mgs(E)}$ for a most general solution.
\end{itemize}


                                \end{frame}

%---------------------------------------------------------------------


                          \begin{frame}\frametitle{Exercise}

Consider a trivial system of equations $\M{\setof{}}$ or
$\M{\setof{a=a}}$.

What is the set of solutions to it?
%any substitution

What is the set of most general solutions to it?
% all renaming substitutions (permutations): x->y, y->x
%the unification algorithm computes {} as the mgu of {a=a}. A renaming
%of the mgu is also an mgu
%the algorithm computes one mgu, all mgus are obtained by applying a
%renaming to any single mgu

%{} is the only idempotent mgu of a trivial system
%{} is the only mgu with isolated equalities of a trivial system

% a substitution σ is idempotent if (σ o σ) = σ, which is the same as saying that domain(σ) ∩ range(σ) = Ø



                                \end{frame}

%---------------------------------------------------------------------


    %                      \begin{frame}\frametitle{Properties}

%\textbf{\OliveGreen{Theorem}}
%Let $\M C$ be a clause and $\M E$ a set of equations. Then

 % \[\M{
  %  \setof{D \in C^* \mid \exists \theta(C\theta = D \text{ and } \theta 
   % \text{ is a solution to } E)} = ((C )\mgs(E))^*.
  %}\]

%In other words, \alert{to find a set of ground instances of a clause
%$\M C$ that also satisfy an equation $\M E$, take the most general
%solution $\sigma$ of $\M E$ and use ground instances of $\M{C \sigma}$.}

 %                               \end{frame}

%---------------------------------------------------------


              	   \begin{frame}\frametitle{Revisit: What should we lift?}

\begin{itemize}
\item Ordering $\M{\succ}$;
\item Selection function $\M{\sel}$;
\item Calculus $\M{\SUPiss}$ (thanks to lifting lemmas).
\end{itemize}

Most importantly, for the lifting to work we use \alert{most general unifiers.}

                                \end{frame}


\end{document}

\section{Saturation and Superposition in Practice}
\input{basic_practice}


\section{Summary of First-Order Theorem proving}
\begin{frame}

Revisiting the example about a commutative group

\end{frame}

\end{document}

\section{Cookies}
\input{cookies}


%\section{Satisfiability Checking}
%\input{satisfiability}

\end{document}




