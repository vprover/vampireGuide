\renewcommand{\eql}{=}

%---------------------------------------------------------------------

                       \begin{frame}\frametitle{First-order logic with equality}

\begin{itemize}
  \item \Blue{Equality predicate}: $\M{\eql}$.
  \item \Blue{Equality}: $\M{l \eql r}$.
\end{itemize}
The order of literals in equalities does not matter, that is, we
consider an equality $\M{l\eql r}$ as a multiset consisting of
two terms $\M{l,r}$, and so consider $\M{l\eql r}$ and $\M{r\eql l}$
equal.

                                \end{frame}

%---------------------------------------------------------------------


                  \begin{frame}\frametitle{Equality. An Axiomatisation
                    (Recap)}


\begin{itemize}
\item \DI{reflexivity}{reflexivity axiom} axiom: $\M{x \eql x}$;

\item \DI{symmetry}{symmetry axiom} axiom: $\M{x \eql y \imply y \eql x}$;

\item \DI{transitivity}{transitivity axiom} axiom: $\M{x \eql y \andl y
  \eql z \imply x \eql z}$;

\item \DI{function substitution (congruence)}{function substitution axioms} axioms:
  $\M{x_1 \eql y_1 \andl \ldots \andl x_n \eql y_n \imply f(\xone{x}{n}) \eql
  f(\xone{y}{n})}$, for every function symbol $\M{f}$;

\item \DI{predicate substitution (congruence)}{predicate substitution axioms} axioms:
  $\M{x_1 \eql y_1 \andl \ldots \andl x_n \eql y_n \andl P(\xone{x}{n}) \imply
  P(\xone{y}{n})}$ for every predicate symbol $\M{P}$.
\end{itemize}

                                \end{frame}


%---------------------------------------------------------------------


      \begin{frame}\frametitle{Inference systems for logic with equality}

We will define a \alert{resolution and superposition inference
  system}. This 
system is \Blue{complete}. One can \Blue{eliminate redundancy}. 
\bigskip


\vs<2->{
  We will first define it only for \Blue{ground clauses}. On the
  theoretical side,

  \begin{itemize}
  \item Completeness is first proved for \Blue{ground clauses} only.\\[.5em]

  \item It is then ``lifted'' to \Blue{arbitrary first-order clauses} using a technique
    called \Blue{lifting}.\\[.5em]

  \item Moreover, this way some notions 
    (ordering, selection function) can first be defined for ground
    clauses only and then it is relatively easy to see how to
    generalise them for non-ground clauses.
  \end{itemize}
}
                           \end{frame}

%---------------------------------------------------------------------


      \begin{frame}\frametitle{Simple Ground Superposition Inference System}

\alert{\underline{Superposition:}} (right and left)

  \M{\[
    \infer[(\ruleSup),]{
      s[r] \eql t \orl C \orl D
      }{
      \WildStrawberry{l \eql r} \orl C
      &
      \WildStrawberry{s[l] \eql t} \orl D
    }
    ~~~
    \infer[(\ruleSup),]{
      s[r] \neql t \orl C \orl D
      }{
      \WildStrawberry{l \eql r} \orl C
      &
      \WildStrawberry{s[l] \neql t} \orl D
    }
  \]}

\vs<2->{
\alert{\underline{Equality Resolution:}} 

  \M{\[
    \infer[(\ruleER),]{
      C
      }{
      \WildStrawberry{s \neql s} \orl C
    }
  \]}
}

\vs<3->{
\alert{\underline{Equality Factoring:}}

  \M{\[
    \infer[(\ruleEF),]{
      s \eql t \orl t \neql t' \orl C
      }{
      \WildStrawberry{s \eql t} \orl \WildStrawberry{s \eql t'} \orl C
    }
  \]}
}
                                \end{frame}

%---------------------------------------------------------------------


                            \begin{frame}\frametitle{Example}

  \M{\[
  \begin{array}{l}
    f(a) \eql a \orl g(a) \eql a \\
    f(f(a)) \eql a \orl g(g(a)) \neql a \\     
    f(f(a)) \neql a     
  \end{array}
  \]}

                                \end{frame}

%---------------------------------------------------------------------


                     \begin{frame}\frametitle{Can this system be used for
                         efficient theorem proving?}

Not really. It has \alert{too many inferences}. For example, from
the clause $\M{f(a) \eql a}$ we can derive any clause of the form

\[ \M{f^m(a) \eql f^n(a)} \]

where $\M{m,n \geq 0}$. 

\vs<2->{
  Worst of all, the derived clauses can be \alert{much larger} than the
  original clause $\M{f(a) \eql a}$.
}

\vs<3->{
  The recipe is to use the previously introduced ingredients:

  \begin{enumerate}
  \item Ordering;
  \item Literal selection;
  \item Redundancy elimination.
  \end{enumerate}
}

                                \end{frame}
\endinput

%---------------------------------------------------------------------


           \begin{frame}\frametitle{Atom and literal orderings on equalities}

Equality atom comparison treats an equality $s \eql t$ as the multiset
$\mssetof{s,t}$.

\begin{itemize}
\item \alert{$(s' \eql t') \succl (s \eql t)$} if
      $\M{\mssetof{s',t'} \succ \mssetof{s,t}}$
\item \alert{$(s' \neql t') \succl (s \neql t)$} if
      $\M{\mssetof{s',t'} \succ \mssetof{s,t}}$
\end{itemize}

%Finally, we assert that \OliveGreen{all non-equality literals be greater
%than all equality literals}.

with $\succl$  being an induced ordering on literals. 

                                \end{frame}


%---------------------------------------------------------------------


      \begin{frame}\frametitle{Ground Superposition Inference System $\M{\SUPiss}$}

\small Let $\M\sigma$ be a well-behaved literal selection function. 

\alert{\underline{Superposition:}} (right and left)
  \M{\[
    \infer[(\ruleSup),]{
      s[r] \eql t \orl C \orl D
      }{
      \WildStrawberry{\underline{l \eql r}} \orl C
      &
      \WildStrawberry{\underline{s[l] \eql t}} \orl D
    }
    ~~~
    \infer[(\ruleSup),]{
      s[r] \neql t \orl C \orl D
      }{
      \WildStrawberry{\underline{l \eql r}} \orl C
      &
      \WildStrawberry{\underline{s[l] \neql t}} \orl D
    }
  \]}%
where (i) $\M{l \succ r}$, (ii) $\M{s[l] \succ t}$ \vs<2->{, (iii)
$\M{\M{l \eql r}}$ is strictly greater than any literal in $\M C$,\\
(iv) (only for the superposition-right rule) $\M{\M{s[l] \eql t}}$ is greater than or equal to
any literal in $\M D$. }


\vs<3->{
\alert{\underline{Equality Resolution:}} 

  \M{\[
    \infer[(\ruleER),]{
      C
      }{
      \WildStrawberry{\underline{s \neql s}} \orl C
    }
  \]}
}

\vs<4->{
\alert{\underline{Equality Factoring:}}

  \M{\[
    \infer[(\ruleEF),]{
      s \eql t \orl t \neql t' \orl C
      }{
      \WildStrawberry{\underline{s \eql t}} \orl s \eql t' \orl C
    }
  \]}
where (i) $\M{s \succ t \succeq t'}$; (ii) $\M{\M{s \eql t}}$ is greater
than or equal to any literal in $\M C$.
}

                                \end{frame}


%---------------------------------------------------------------------

         \begin{frame}\frametitle{Extension to arbitrary (non-equality) literals}

\begin{itemize}
\item Consider a \Blue{two-sorted logic} in which equality is the only predicate
symbol.

\item Interpret terms as terms of the first sort and \Blue{non-equality atoms
as terms of the second sort}.

\item Add a \alert{constant $\M\top$ of the second sort}.

\item Replace \Blue{non-equality atoms $\M{p(\xone{t}{n})}$ by
equalities of the second sort $\M{p(\xone{t}{n}) \eql \top}$}.
\end{itemize}

\vs<2->{
For example, the clause

  \[
    p(a,b) \orl \notl q(a) \orl a \neq b
  \]
becomes

  \[
    p(a,b) \eql \top \orl q(a) \neql \top \orl a \neq b.
  \]
}

                     \end{frame}

%---------------------------------------------------------------------

             \begin{frame}\frametitle{Binary resolution inferences can be
                 represented by inferences in the superposition system}

We ignore selection functions.

  \[\M{
      \infer[(\BRr)]{C_1 \orl C_2}{A \orl C_1 & \notl A \orl C_2}}
  \]

  \[\M{
      \infer[(\ruleER)]{
        C_1 \orl C_2
        }{
        \infer[(\ruleSup)]{
          \top \neql \top \orl C_1 \orl C_2
          }{
          A \eql \top \orl C_1 & A \neql \top \orl C_2
        }
      }
  }\]

                     \end{frame}

%---------------------------------------------------------------------

             \begin{frame}\frametitle{Exercise}

\PineGreen{Positive factoring can also be
represented by inferences in the superposition system.}

                     \end{frame}

                     \end{document}%end of lecture 14

\section{Term Orderings}
%---------------------------------------------------------------------

\begin{frame}
\frametitle{Simplification Ordering}


When we deal with equality, we need to work with \alert{term
  orderings}. 

Consider a strict ordering $\M{\succ}$ on signature symbols, such that
$\M{\succ}$ is well-founded. 


%The only restriction we imposed on term orderings was 
%\Blue{well-foundedness} and \Blue{stability under substitutions}.
%When we deal with equality, these two properties are insufficient.
%We need a third property, called \alert{monotonicity}.

The ordering $\M{\succ}$ on terms is called a \alert{simplification ordering}
if

\begin{enumerate}
\item $\M{\succ}$ is \Blue{well-founded};
\item $\M{\succ}$ is \Blue{monotonic}: 
if $\M{l \succ r}$, then $\M{s[l] \succ s[r]}$;
\item $\M{\succ}$ is \Blue{stable under substitutions}: 
if $\M{l \succ r}$, then $\M{l\theta \succ r\theta}$.
\end{enumerate}

\vs<2->{
One can combine the last two properties into one:

\begin{enumerate}
  \item [2a.] If $\M{l \succ r}$, then
  $\M{s[l\theta] \succ s[r\theta]}$.
\end{enumerate}
}
                                \end{frame}

%---------------------------------------------------------------------

                     \begin{frame}\frametitle{A General Property of Term Orderings}

If $\M{\succ}$ is a simplification ordering, then for every term $\M{t[s]}$
and its proper subterm $\M{s}$ we have $\M{s \not \succ t[s]}$. {\Blue{Why?}}

\medskip

\vs<2->{
Consider an example.

  \[\M{
    \begin{array}{l}
      f(a) \eql a \\
      f(f(a)) \eql a \\
      f(f(f(a))) \eql a
    \end{array}}
  \]
Then both $\M{f(f(a)) \eql a}$ and $\M{f(f(f(a))) \eql a}$
are \alert{redundant}. The clause $f(a) \eql a$ is a logical consequence
of $\M{\setof{f(f(a)) \eql a,f(f(f(a))) \eql a}}$ but is \alert{not
redundant}.
}

\bigskip


\vs<3->{\Blue{Exercise: Show that $\M{\{ f(a)=a, f(f(f(a)))\neql a\}}$ is
  unsatisfiable, by using superposition with redundancy elimination}. }

\bigskip

\vs<4->{How to ``come up'' with \alert{simplification orderings}?}
                                \end{frame}

%---------------------------------------------------------------------



                           \begin{frame}\frametitle{Term Algebra}


\alert{Term algebra $\TA(\Sigma)$} of signature $\M{\Sigma}$:

\begin{itemize}
\item \Blue{Domain}: the set of all ground terms of $\M\Sigma$.
\item \Blue{Interpretation of any function symbol $\M f$ or constant
      $\M c$ is defined as follows:}:

  \[\M{
    \begin{array}{rcl}
    f_{\TA(\Sigma)}(\xone{t}{n}) & \bydef & f(\xone{t}{n}); \\
    c_{\TA(\Sigma)} & \bydef & c.
    \end{array}
  }\]
\end{itemize}



                               \end{frame}


%---------------------------------------------------------------------

                     \begin{frame}\frametitle{Knuth-Bendix Ordering (KBO), Ground Case}


\begin{columns}
\column{0.5\textwidth}
Let us fix

\begin{itemize}
  \item Signature $\M{\Sigma}$, it induces the \alert{term algebra} 
  $\M{\TA(\Sigma)}$.

  \item Total ordering $\M{\gg}$ on $\M{\Sigma}$, called \alert{precedence
  relation};

  \item \alert{Weight function} $\M{w: \Sigma \rightarrow \nat}$.\\
\end{itemize}

\vs<2->{\alert{Weight} of a  ground term $t$ is

    \[
       |g(t_1,\ldots,t_n)| =
        w(g)+ \sum_{i=1}^n |t_i|.
   \]
}

\column{0.5\textwidth}
\vs<3->{
\alert{$g(t_1,\ldots,t_n) \KBo h(s_1,\ldots,s_m)$} if
\begin{enumerate}
  \item \vs<4->{$\M{|g(t_1,\ldots,t_n)|>|h(s_1,\ldots,s_m)|}$

  \OliveGreen{(by weight)} or}

  \item \vs<5->{$\M{|g(t_1,\ldots,t_n)|=|h(s_1,\ldots,s_m)|}$ and one of the following
  holds:

  \begin{enumerate}
     \item $\M{g \gg h}$ \OliveGreen{(by precedence)} or
     \item  \vs<6->{$\M{g=h}$ and for some $\M{1 \leq i \leq n}$ we have
     $\M{t_1=s_1,\ldots,t_{i-1}=s_{i-1}}$ and $\M{t_i \KBo s_i}$ 
     \OliveGreen{(lexicographically)}.}
   \end{enumerate}}
\end{enumerate}
}
\end{columns}

                                \end{frame}

%---------------------------------------------------------------------
\begin{frame}
\frametitle{Example}

\[
\begin{array}{rcl}
 w(a) & = & 1 \\
 w(b) & = & 2 \\
 w(f) & = & 3 \\
 w(g) & = & 0 
\end{array}
\]

\[
   |f(g(a),f(a,b))| \vs<2->{= |3(0(1),3(1,2))|} \vs<3->{= 3+0+1+3+1+2} \vs<4->{= 10.}
\]

%\vs<5->{
%  There exists also a \Blue{non-ground version} of the Knuth-Bendix ordering
 % and a (nearly) \alert{linear time algorithm} for term comparison using this
 % ordering. 
%}

\vskip1em

\vs<5->{
  The Knuth-Bendix ordering is the \alert{main ordering} used in Vampire and
  all other resolution and superposition theorem provers.
}

\end{frame}

%---------------------------------------------------------------------
        \begin{frame}\frametitle{Knuth-Bendix Ordering (KBO),
                         Ground Case\vs<1>{{\small : Summary} }}


\begin{columns}
\column{0.5\textwidth}
Let us fix

\begin{itemize}
  \item Signature $\M{\Sigma}$, it induces the \alert{term algebra} 
  $\M{\TA(\Sigma)}$.

  \item Total ordering $\M{\gg}$ on $\M{\Sigma}$, called \alert{precedence
  relation};

  \item \alert{Weight function} $\M{w: \Sigma \rightarrow \nat}$.\\
\end{itemize}

\vs<1->{\alert{Weight} of a  ground term $t$ is

    \[
       |g(t_1,\ldots,t_n)| =
        w(g)+ \sum_{i=1}^n |t_i|.
   \]
}

\column{0.5\textwidth}
\vs<1->{
\alert{$g(t_1,\ldots,t_n) \KBo h(s_1,\ldots,s_m)$} if
\begin{enumerate}
  \item \vs<1->{$\M{|g(t_1,\ldots,t_n)|>|h(s_1,\ldots,s_m)|}$

  \OliveGreen{(by weight)} or}

  \item \vs<1->{$\M{|g(t_1,\ldots,t_n)|=|h(s_1,\ldots,s_m)|}$ and one of the following
  holds:

  \begin{enumerate}
     \item $\M{g \gg h}$ \OliveGreen{(by precedence)} or
     \item  \vs<1->{$\M{g=h}$ and for some $\M{1 \leq i \leq n}$ we have
     $\M{t_1=s_1,\ldots,t_{i-1}=s_{i-1}}$ and $\M{t_i \KBo s_i}$ 
     \OliveGreen{{\small (lexicographically, i.e. left-to-right)}}.}
   \end{enumerate}}
\end{enumerate}
}
\end{columns}

\vspace*{1em}

\vs<2->{Note: \alert{Weight functions} $\M{w}$ are \Blue{\bf not arbitrary
    functions}\\ -- need to be ``compatible'' with $\M{\gg}$.}

\bigskip
\vs<3->{Why? Compare for example $a$ and $f(a)$ with arbitrary $\gg$
  and $w$.}

\end{frame}


 % ---------------------------------------------------------
\begin{frame}{Weight Functions, Ground Case}

    A \alert{weight function} $\M{w: \Sigma \rightarrow \nat}$
    is any function satisfying:

    \begin{itemize}
    \item $w(a)>0$ for any constant $a\in\Sigma$;\\[.5em]
      
    \item if $w(f)=0$ for a unary function $f\in\Sigma$, then $f\gg g$
      for all functions $g\in\Sigma$ with $f\neq g$.\\

      That is, $f$ is the greatest element of $\Sigma$ wrt $\gg$.
      
     \end{itemize}

     \bigskip
As a consequence, there is at most one unary function $f$ with
$w(f)=0$.


\end{frame}

%-----------------------------------------

%

      \begin{frame}\frametitle{Same Property as for $\BRiss$}

The conclusion is \alert{strictly smaller} than the rightmost premise:

 \M{\[
    \infer[(\ruleSup),]{
      s[r] \eql t \orl C \orl D
      }{
      \WildStrawberry{\underline{l \eql r}} \orl C
      &
      \WildStrawberry{\underline{s[l] \eql t}} \orl D
    }
    ~~~
    \infer[(\ruleSup),]{
      s[r] \neql t \orl C \orl D
      }{
      \WildStrawberry{\underline{l \eql r}} \orl C
      &
      \WildStrawberry{\underline{s[l] \neql t}} \orl D
    }
  \]}%
where (i) $\M{l \succ r}$, (ii) $\M{s[l] \succ t}$, (iii)
$\M{\M{l \eql r}}$ is strictly greater than any literal in $\M C$,
(iv) $\M{\M{s[l] \eql t}}$ is greater than or equal to
any literal in $\M D$.


                               \end{frame}

%---------------------------------------------------------------------


      \begin{frame}\frametitle{New redundancy}

Consider a superposition with a unit left premise:

 \[
    \infer[(\ruleSup),]{
      s[r] \eql t \orl D
      }{
      \underline{l \eql r}
      &
      \underline{s[l] \eql t} \orl D
    }
 \]

Note that we have

 \[
     l \eql r,
      s[r] \eql t \orl D
     \alert{\models}
      s[l] \eql t \orl D
\]

\vs<2->{
and we have

 \[
 s[l] \eql t \orl D
 \alert{\succ}
 s[r] \eql t \orl D.
 \]
}

\vs<3->{
If we also have $s[l] \eql t \orl D \succ l\eql r$, then the second
premise is \alert{redundant} and can be removed.
}

\medskip

\vs<4->{
  This rule (superposition plus deletion) is sometimes called
  \alert{demodulation} (also \alert{rewriting by unit equalities}).
}


                               \end{frame}



%---------------------------------------------------------------------
\begin{frame}
\frametitle{Exercise}
Consider the KBO ordering $\succ$ generated by:\\[.5em]

-- the precedence $ P\gg Q\gg f\gg a$;\\[.5em]

and\\[.5em]

-- the weight function $w$ with $w(P)=w(Q)=2$, $w(f)=w(a)=1$.\\[1em]


Consider the set of clauses $S$ to be: 
\[\begin{array}{l}
{Q(a)},\\
 {\neg Q(a)}\vee f(a)=a, \\ 
 {\neg P(a)},\\ {P(f(a))}\}.
\end{array}\] 

Apply saturation on $S$ by using an inferece process with redundancy
based on the (ground) superposition calculus $\SUPiss$. 
\end{frame}





