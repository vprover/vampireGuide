%---------------------------------------------------------------------

                        \begin{frame}\frametitle{Inference System}

\begin{itemize}
\item \alert{inference} has the form

  \M{\[
     \infer[,]{G}{F_1 & \ldots & F_n}
  \]}
where $\M{n \geq 0}$ and $\M{\xone{F}{n},G}$ are formulas. 
\item
The formula $\M{G}$ is called the \alert{conclusion} of the inference;
\item
The formulas $\M{\xone{F}{n}}$ are called its \DI{premises}{premise}.
\item
An \DII{inference rule}{inference rule}{rule!inference} $\M{R}$ is a set of
inferences. 
\item
Every inference $\M{I \in R}$ is called an 
\DI{instance of $\M{R}$}{instance!of inference rule}.
\item
An \DII{Inference system}{inference system}{system!inference} $\M{\isI}$
is a set of inference rules. 
\item \alert{Axiom:} inference rule with no premises.
\end{itemize}

                                \end{frame}

%---------------------------------------------------------------------

                        \begin{frame}\frametitle{Inference System: Example}

Represent the natural number $n$ by the string $\underbrace{| \ldots
  |}_{n~\mathrm{times}}\varepsilon$.

The following inference system contains 6 inference rules for deriving
equalities between expressions containing natural numbers, addition
$+$ and multiplication $\cdot$.

  \[
    \infer[(\varepsilon)]{\varepsilon = \varepsilon}{}
    ~~~~~~~~~~
    \infer[(|)]{|x = |y}{x = y}
 \]

  \[
    \infer[(+_1)]{\varepsilon + x = x}{}
    ~~~~~~~~~~
    \infer[(+_2)]{|x + y = |z}{x + y = z}
  \]

  \[
    \infer[(\cdot_1)]{\varepsilon \cdot x = \varepsilon}{}
    ~~~~~~~~~~
    \infer[(\cdot_2)]{|x \cdot y = z}{x \cdot y = u & y + u = z}
  \]


                               \end{frame}

%---------------------------------------------------------------------

                        \begin{frame}\frametitle{Derivation, Proof}


\begin{itemize}
\item \DI{Derivation}{derivation} in an inference system $\M{\isI}$: a tree
built from inferences in $\M{\isI}$.

\item
If the root of this derivation is $\M{E}$, then we say it is a
\alert{derivation of $\M{E}$}.

\item \DI{Proof}{proof} of $\M{E}$: a finite derivation whose leaves 
are axioms.

\item \alert{Derivation of $\M{E}$ from $\M{\xone{E}{m}}$:}
a finite derivation of $\M{E}$ whose every leaf is either
an axiom or one of the expressions $\M{\xone{E}{m}}$.
\end{itemize}

                               \end{frame}

%---------------------------------------------------------------------

                        \begin{frame}\frametitle{Examples}

For example,
 \[
  \infer[(+_2)]{|||\varepsilon + |\varepsilon = ||||\varepsilon}{
    ||\varepsilon + |\varepsilon = |||\varepsilon}
  \]

is an \Blue{inference} that is an instance (special case) of the
\Blue{inference rule}

  \[
  \infer[(+_2)]{|x + y = |z}{x + y = z}
  \]

\vs<2->{
  It has one \Blue{premise} $||\varepsilon + |\varepsilon = |||\varepsilon$
  and the \Blue{conclusion} $|||\varepsilon + |\varepsilon = ||||\varepsilon$.
}

\medskip

\vs<3->{
  The \Blue{axiom}

  \[
  \infer[(+_1)]{\varepsilon + |||\varepsilon = |||\varepsilon}{}
  \]

  is an instance of the rule

  \[
  \infer[(+_1)]{\varepsilon + x = x}{}
  \]
}

                               \end{frame}

%---------------------------------------------------------------------

                    \begin{frame}\frametitle{Proof, \vs<2->{Derivation} in this Inference System}

Proof of $\alert<1>{||\varepsilon \cdot ||\varepsilon = ||||\varepsilon}$
(that is, $2 \cdot 2 = 4$).\\[1em]

\vs<2->{Derivation of $\alert{|\varepsilon \cdot ||\varepsilon =
    ||\varepsilon}$ from $\varepsilon \cdot ||\varepsilon = \varepsilon$
  and $|\varepsilon + \varepsilon = |\varepsilon$. }

\bigskip

  \[\M{
    \infer[(\cdot_2).]{\alert<1>{||\varepsilon \cdot ||\varepsilon = ||||\varepsilon}}{
      \infer[(\cdot_2)]{\alert<2->{|\varepsilon \cdot ||\varepsilon = ||\varepsilon}}{
        \infer[(\cdot_1)]{\alert<2->{\varepsilon \cdot ||\varepsilon = \varepsilon}}{}
        &
        \infer[(+_2)]{\alert<2->{||\varepsilon + \varepsilon = ||\varepsilon}}{
          \infer[(+_2)]{\alert<2->{|\varepsilon + \varepsilon = |\varepsilon}}{
            \infer[(+_1)]{\varepsilon + \varepsilon = \varepsilon}{}
          }
        }
      }
      & 
      \infer[(+_2)]{||\varepsilon + ||\varepsilon = ||||\varepsilon}{
        \infer[(+_2)]{|\varepsilon + ||\varepsilon = |||\varepsilon}{
          \infer[(+_1)]{\varepsilon + ||\varepsilon = ||\varepsilon}{}
        }
      }
    }
  }\]


                           \end{frame}

%---------------------------------------------------------------------

                      \begin{frame}
            \frametitle{Arbitrary First-Order Formulas}

\begin{itemize}
\item A \alert{first-order signature (vocabulary)}: function symbols (including 
  constants), predicate symbols. \alert{Equality} is part of the
  language.

\item A set of \alert{variables}.

\item \alert{Terms} are buit using variables and function symbols. For
  example, $f(x) + g(x)$.

\item \alert{Atoms}, or \alert{atomic formulas} are obtained by
  applying a predicate symbol to a sequence of terms. For example,
  $p(a,x)$ or $f(x) + g(x) \geq 2$.

\item \alert{Formulas:} built from atoms using logical
    connectives $\notl$, $\andl$, $\orl$, $\implies$, $\iffl$ and
    quantifiers $\forall$, $\exists$. For example,
  $(\forall x)x = 0 \orl (\exists y)y > x$.
\end{itemize}


                                \end{frame}
%---------------------------------------------------------------------

                      \begin{frame}
            \frametitle{Clauses}

\begin{itemize}
\item \alert{Literal:} either an atom $A$ or its
    negation $\notl A$.

\item \alert{Clause:} a disjunction $L_1 \orl \ldots \orl L_n$ of
literals, where $n \geq 0$.

\item \visible<2->{\alert{Empty clause}, denoted by $\alert{\emptyclause}$: clause
with 0 literals, that is, when $n = 0$.}

\item \visible<3->{A formula in \alert{Clausal Normal Form (CNF)}: a
    conjunction of clauses.}

\item \visible<4->{From now on: A clause is \alert{ground} if it contains no variables.}

\item \visible<4->{If a clause contains variables, we assume that it \alert{implicitly
  universally quantified}. That is, we treat $p(x) \orl q(x)$ as
$\forall x(p(x) \orl q(x))$.}  
\end{itemize}


                                \end{frame}

%---------------------------------------------------------------------

              	   \begin{frame}
           \frametitle{Binary Resolution Inference System}

The \DII{binary resolution inference system}{binary 
resolution!inference system}{inference system!binary resolution},
denoted by \alert{$\BRis$} is an inference system on \Blue{propositional}
clauses (or \Blue{ground} clauses).

It consists of two inference rules:

\begin{itemize}
\item
  \alert{Binary resolution}, denoted by $\alert{\BRr}$:

  \[
      \infer[(\BRr).]{C_1 \orl C_2}{p \orl C_1 & \notl p \orl C_2}
  \]

\item
  \alert{Factoring}, denoted by $\alert{\Fact}$:

  \[
     \infer[(\Fact).]{L \orl C}{L \orl L \orl C}
 \]
\end{itemize}



                           \end{frame}

%---------------------------------------------------------------------

	   \begin{frame}\frametitle{Soundness}

\begin{itemize}
\item
  \alert{An inference is sound} if 
  the conclusion of this inference is a logical 
  consequence of its premises.

\item
  \alert{An inference system
  is sound} if every inference rule in this system is sound.
\end{itemize}

\bigskip

\vs<2->{
\Blue{$\M{\BRis}$ is sound.}

\bigskip

\vs<3->{
Consequence of soundness: let $S$ be a set of clauses. If
$\emptyclause$ can be derived from $S$ in $\BRis$, then
$S$ is \alert{unsatisfiable}.}

}


                           \end{frame}

%---------------------------------------------------------------------

	   \begin{frame}\frametitle{Example}

Consider the following set $S$ of clauses

    \[\M{
      \setof{\notl p \orl \notl q,~\notl p \orl q,~p \orl \notl q,~p \orl q}.
    }\]


  Is $S$ unsatisfiable?

  \bigskip

\only<2->{

The following derivation derives
the empty clause from this set:

    \[\M{
      \infer[(\BRr)]{\emptyclause}{
        \infer[(\Fact)]{p}{
          \infer[(\BRr)]{p \orl p}{
            p \orl q &
	    p \orl \notl q
          }
        }
        &
        \infer[(\Fact)]{\notl p}{
          \infer[(\BRr)]{\notl p \orl \notl p}{
            \notl p \orl q &
	    \notl p \orl \notl q
          }
        }
      }
    }\]
  Hence, this set $S$  of clauses is \alert{unsatisfiable}.
  }

                           \end{frame}


%---------------------------------------------------------------------

	   \begin{frame}\frametitle{Exercise}

Consider the following set $S$ of clauses

    \[\M{
      \setof{\notl p \orl \notl q,~\notl p \orl q,~p \orl \notl q,~p \orl q}.
    }\]


Show that there exists an infinite number of different $\M{\BRis}$
derivations of  the empty clause $\emptyclause$ from the clauses of
$S$. 

\end{frame}

% ---------------------------------------------------------------------
\begin{frame}\frametitle{Soundness - Summarized}


\begin{itemize}
\item
  \alert{An inference is sound} if 
  the conclusion of this inference is a logical 
  consequence of its premises.

\item
  \alert{An inference system
  is sound} if every inference rule in this system is sound.
\end{itemize}

\bigskip

\vs<1->{
\Blue{$\M{\BRis}$ is sound.}

\bigskip

\vs<1->{
Consequence of soundness: let $S$ be a set of clauses. If
$\emptyclause$ can be derived from $S$ in $\BRis$, then
$S$ is \alert{unsatisfiable}.}

}


                           \end{frame}
%---------------------------------------------------------------------

                           
	   \begin{frame}
\frametitle{Can this be used for checking (un)satisfiability?}


\begin{enumerate}
  \item What happens when the empty clause \alert{cannot be derived}
    from $\M{S}$?

    \bigskip
    
  \item \alert{How} can one search for possible derivations of the empty clause?
\end{enumerate}

                           \end{frame}

%---------------------------------------------------------------------

	   \begin{frame}
\frametitle{Can this be used for checking (un)satisfiability?}

\begin{enumerate}\item
\alert{Completeness.}

\begin{quotation}\noindent\Blue{%
Let $\M{S}$ be an unsatisfiable set of clauses. Then there exists a 
derivation of $\M{\emptyclause}$ from $\M{S}$ in $\M{\BRis}$.}
\end{quotation}

\bigskip

\item \vs<2->{
We have to formalize \alert{search for derivations.}}
\end{enumerate}
\medskip 
\vs<2->{
  However, before doing this we will introduce a slightly more refined
  inference system.
}

                           \end{frame}


