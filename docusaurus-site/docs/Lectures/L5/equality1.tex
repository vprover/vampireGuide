\renewcommand{\eql}{=}


      \begin{frame}\frametitle{Simple Ground Superposition Inference System}

\alert{\underline{Superposition:}} (right and left)

  \M{\[
    \infer[(\ruleSup),]{
      s[r] \eql t \orl C \orl D
      }{
      \WildStrawberry{l \eql r} \orl C
      &
      \WildStrawberry{s[l] \eql t} \orl D
    }
    ~~~
    \infer[(\ruleSup),]{
      s[r] \neql t \orl C \orl D
      }{
      \WildStrawberry{l \eql r} \orl C
      &
      \WildStrawberry{s[l] \neql t} \orl D
    }
  \]}

\vs<1->{
\alert{\underline{Equality Resolution:}} 

  \M{\[
    \infer[(\ruleER),]{
      C
      }{
      \WildStrawberry{s \neql s} \orl C
    }
  \]}
}

\vs<2->{
\alert{\underline{Equality Factoring:}}

  \M{\[
    \infer[(\ruleEF),]{
      s \eql t \orl t \neql t' \orl C
      }{
      \WildStrawberry{s \eql t} \orl \WildStrawberry{s \eql t'} \orl C
    }
  \]}
}
                                \end{frame}

%---------------------------------------------------------------------


                     \begin{frame}\frametitle{Can this system be used for
                         efficient theorem proving?}

Not really. It has \alert{too many inferences}. For example, from
the clause $\M{f(a) \eql a}$ we can derive any clause of the form

\[ \M{f^m(a) \eql f^n(a)} \]

where $\M{m,n \geq 0}$.

\bigskip

\vs<1->{
  Worst of all, the derived clauses can be \alert{much larger} than the
  original clause $\M{f(a) \eql a}$.
}

\bigskip

\vs<2->{
  The recipe is to use the previously introduced ingredients:

  \begin{enumerate}
  \item Ordering;
  \item Literal selection;
  \item Redundancy elimination.
  \end{enumerate}
}

                                \end{frame}


%---------------------------------------------------------------------


           \begin{frame}\frametitle{Atom and literal orderings on equalities}

Equality atom comparison treats an equality $s \eql t$ as the multiset
$\mssetof{s,t}$.

\begin{itemize}
\item \alert{$(s' \eql t') \succl (s \eql t)$} if
      $\M{\mssetof{s',t'} \succ \mssetof{s,t}}$
\item \alert{$(s' \neql t') \succl (s \neql t)$} if
      $\M{\mssetof{s',t'} \succ \mssetof{s,t}}$
\end{itemize}

%Finally, we assert that \OliveGreen{all non-equality literals be greater
%than all equality literals}.

with $\succl$  being an induced ordering on literals. 

                                \end{frame}


%---------------------------------------------------------------------


      \begin{frame}\frametitle{Ground Superposition Inference System $\M{\SUPiss}$}

\small Let $\M\sigma$ be a well-behaved literal selection function. 

\alert{\underline{Superposition:}} (right and left)
  \M{\[
    \infer[(\ruleSup),]{
      s[r] \eql t \orl C \orl D
      }{
      \WildStrawberry{\underline{l \eql r}} \orl C
      &
      \WildStrawberry{\underline{s[l] \eql t}} \orl D
    }
    ~~~
    \infer[(\ruleSup),]{
      s[r] \neql t \orl C \orl D
      }{
      \WildStrawberry{\underline{l \eql r}} \orl C
      &
      \WildStrawberry{\underline{s[l] \neql t}} \orl D
    }
  \]}%
where (i) $\M{l \succ r}$, (ii) $\M{s[l] \succ t}$ \vs<1->{, (iii)
$\M{\M{l \eql r}}$ is strictly greater than any literal in $\M C$,\\
(iv) (only for the superposition-right rule) $\M{\M{s[l] \eql t}}$ is greater than or equal to
any literal in $\M D$. }

\smallskip

\vs<1->{
\alert{\underline{Equality Resolution:}} 

  \M{\[
    \infer[(\ruleER),]{
      C
      }{
      \WildStrawberry{\underline{s \neql s}} \orl C
    }
  \]}
}

\smallskip

\vs<1->{
\alert{\underline{Equality Factoring:}}

  \M{\[
    \infer[(\ruleEF),]{
      s \eql t \orl t \neql t' \orl C
      }{
      \WildStrawberry{\underline{s \eql t}} \orl s \eql t' \orl C
    }
  \]}
where (i) $\M{s \succ t \succeq t'}$; (ii) $\M{\M{s \eql t}}$ is greater
than or equal to any literal in $\M C$.
}

                                \end{frame}


                                

%---------------------------------------------------------------------

         \begin{frame}\frametitle{Extension to arbitrary (non-equality) literals}

\begin{itemize}
\item Consider a \Blue{two-sorted logic} in which equality is the only predicate
symbol.

\item Interpret terms as terms of the first sort and \Blue{non-equality atoms
as terms of the second sort}.

\item Add a \alert{constant $\M\top$ of the second sort}.

\item Replace \Blue{non-equality atoms $\M{p(\xone{t}{n})}$ by
equalities of the second sort $\M{p(\xone{t}{n}) \eql \top}$}.
\end{itemize}

\vs<2->{
For example, the clause

  \[
    p(a,b) \orl \notl q(a) \orl a \neq b
  \]
becomes

  \[
    p(a,b) \eql \top \orl q(a) \neql \top \orl a \neq b.
  \]
}

                     \end{frame}

%---------------------------------------------------------------------

             \begin{frame}\frametitle{Binary resolution inferences can be
                 represented by inferences in the superposition system}

We ignore selection functions.

  \[\M{
      \infer[(\BRr)]{C_1 \orl C_2}{A \orl C_1 & \notl A \orl C_2}}
  \]

  \[\M{
      \infer[(\ruleER)]{
        C_1 \orl C_2
        }{
        \infer[(\ruleSup)]{
          \top \neql \top \orl C_1 \orl C_2
          }{
          A \eql \top \orl C_1 & A \neql \top \orl C_2
        }
      }
  }\]

                     \end{frame}

%---------------------------------------------------------------------

             \begin{frame}\frametitle{Exercise}

\PineGreen{Positive factoring can also be
represented by inferences in the superposition system.}

                     \end{frame}

    

