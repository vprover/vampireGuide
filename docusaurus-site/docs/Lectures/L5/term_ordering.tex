%\renewcommand{\eql}{=}
 
%---------------------------------------------------------------------

\begin{frame}
\frametitle{Simplification Ordering}


When we deal with equality, we need to work with \alert{term
  orderings}. 

Consider a strict ordering $\M{\succ}$ on signature symbols, such that
$\M{\succ}$ is well-founded. 

\medskip
%The only restriction we imposed on term orderings was 
%\Blue{well-foundedness} and \Blue{stability under substitutions}.
%When we deal with equality, these two properties are insufficient.
%We need a third property, called \alert{monotonicity}.

The ordering $\M{\succ}$ on terms is called a \alert{simplification ordering}
if

\begin{enumerate}
\item $\M{\succ}$ is \Blue{well-founded};
\item $\M{\succ}$ is \Blue{monotonic}: 
if $\M{l \succ r}$, then $\M{s[l] \succ s[r]}$;
\item $\M{\succ}$ is \Blue{stable under substitutions}: 
if $\M{l \succ r}$, then $\M{l\theta \succ r\theta}$.
\end{enumerate}

\medskip 
\vs<2->{
One can combine the last two properties into one:

\begin{enumerate}
  \item [2a.] If $\M{l \succ r}$, then
  $\M{s[l\theta] \succ s[r\theta]}$.
\end{enumerate}
}
                                \end{frame}

%---------------------------------------------------------------------

                     \begin{frame}\frametitle{A General Property of Term Orderings}

If $\M{\succ}$ is a simplification ordering, then for every term $\M{t[s]}$
and its proper subterm $\M{s}$ we have $\M{s \not \succ t[s]}$. {\Blue{Why?}}

\medskip

\vs<2->{
Consider an example.

  \[\M{
    \begin{array}{l}
      f(a) \eql a \\
      f(f(a)) \eql a \\
      f(f(f(a))) \eql a
    \end{array}}
  \]
Then both $\M{f(f(a)) \eql a}$ and $\M{f(f(f(a))) \eql a}$
are \alert{redundant}.

\vs<3->{The clause $f(a) \eql a$ is a logical consequence
of $\M{\setof{f(f(a)) \eql a,f(f(f(a))) \eql a}}$ but is \alert{not
redundant}.}
}

\bigskip


\vs<3->{\Blue{Exercise: Show that $\M{\{ f(a)=a, f(f(f(a)))\neql a\}}$ is
  unsatisfiable, by using superposition with redundancy elimination}. }

\bigskip

\vs<4->{How to ``come up'' with \alert{simplification orderings}?}
                                \end{frame}

%---------------------------------------------------------------------



                           \begin{frame}\frametitle{Term Algebra}


\alert{Term algebra $\TA(\Sigma)$} of signature $\M{\Sigma}$:

\begin{itemize}
\item \Blue{Domain}: the set of all ground terms of $\M\Sigma$.
\item \Blue{Interpretation of any function symbol $\M f$ or constant
      $\M c$ is defined as:}

  \[\M{
    \begin{array}{rcl}
    f_{\TA(\Sigma)}(\xone{t}{n}) & \bydef & f(\xone{t}{n}); \\
    c_{\TA(\Sigma)} & \bydef & c.
    \end{array}
  }\]
\end{itemize}



                               \end{frame}


%---------------------------------------------------------------------

                     \begin{frame}\frametitle{Knuth-Bendix Ordering (KBO), Ground Case}


\begin{columns}
\column{0.5\textwidth}
Let us fix

\begin{itemize}
  \item Signature $\M{\Sigma}$, it induces the \alert{term algebra} 
  $\M{\TA(\Sigma)}$.

  \item Total ordering $\M{\gg}$ on $\M{\Sigma}$, called \alert{precedence
  relation};

  \item \alert{Weight function} $\M{w: \Sigma \rightarrow \nat}$.\\
\end{itemize}

\vs<2->{\alert{Weight} of a  ground term $t$ is

    \[
       |g(t_1,\ldots,t_n)| =
        w(g)+ \sum_{i=1}^n |t_i|.
   \]
}

\column{0.5\textwidth}
\vs<3->{
\alert{$g(t_1,\ldots,t_n) \KBo h(s_1,\ldots,s_m)$} if
\begin{enumerate}
  \item \vs<4->{$\M{|g(t_1,\ldots,t_n)|>|h(s_1,\ldots,s_m)|}$

  \OliveGreen{(by weight)} or}

  \item \vs<5->{$\M{|g(t_1,\ldots,t_n)|=|h(s_1,\ldots,s_m)|}$ and one of the following
  holds:

  \begin{enumerate}
     \item $\M{g \gg h}$ \OliveGreen{(by precedence)} or
     \item  \vs<6->{$\M{g=h}$ and for some $\M{1 \leq i \leq n}$ we have
     $\M{t_1=s_1,\ldots,t_{i-1}=s_{i-1}}$ and $\M{t_i \KBo s_i}$ 
     \OliveGreen{(lexicographically)}.}
   \end{enumerate}}
\end{enumerate}
}
\end{columns}

                                \end{frame}

%---------------------------------------------------------------------
\begin{frame}
\frametitle{Example}

\[
\begin{array}{rcl}
 w(a) & = & 1 \\
 w(b) & = & 2 \\
 w(f) & = & 3 \\
 w(g) & = & 0 
\end{array}
\]

\[
   |f(g(a),f(a,b))| \vs<2->{= |3(0(1),3(1,2))|} \vs<3->{= 3+0+1+3+1+2} \vs<4->{= 10.}
\]

%\vs<5->{
%  There exists also a \Blue{non-ground version} of the Knuth-Bendix ordering
 % and a (nearly) \alert{linear time algorithm} for term comparison using this
 % ordering. 
%}

\vskip1em

\vs<5->{
  The Knuth-Bendix ordering is the \alert{main ordering} used in Vampire and
  all other resolution and superposition theorem provers.
}

\end{frame}

%---------------------------------------------------------------------
        \begin{frame}\frametitle{Knuth-Bendix Ordering (KBO),
                         Ground Case\vs<1>{{\small : Summary} }}


\begin{columns}
\column{0.5\textwidth}
Let us fix

\begin{itemize}
  \item Signature $\M{\Sigma}$, it induces the \alert{term algebra} 
  $\M{\TA(\Sigma)}$.

  \item Total ordering $\M{\gg}$ on $\M{\Sigma}$, called \alert{precedence
  relation};

  \item \alert{Weight function} $\M{w: \Sigma \rightarrow \nat}$.\\
\end{itemize}

\vs<1->{\alert{Weight} of a  ground term $t$ is

    \[
       |g(t_1,\ldots,t_n)| =
        w(g)+ \sum_{i=1}^n |t_i|.
   \]
}

\column{0.5\textwidth}
\vs<1->{
\alert{$g(t_1,\ldots,t_n) \KBo h(s_1,\ldots,s_m)$} if
\begin{enumerate}
  \item \vs<1->{$\M{|g(t_1,\ldots,t_n)|>|h(s_1,\ldots,s_m)|}$

  \OliveGreen{(by weight)} or}

  \item \vs<1->{$\M{|g(t_1,\ldots,t_n)|=|h(s_1,\ldots,s_m)|}$ and one of the following
  holds:

  \begin{enumerate}
     \item $\M{g \gg h}$ \OliveGreen{(by precedence)} or
     \item  \vs<1->{$\M{g=h}$ and for some $\M{1 \leq i \leq n}$ we have
     $\M{t_1=s_1,\ldots,t_{i-1}=s_{i-1}}$ and $\M{t_i \KBo s_i}$ 
     \OliveGreen{{\small (lexicographically, i.e. left-to-right)}}.}
   \end{enumerate}}
\end{enumerate}
}
\end{columns}

\vspace*{1em}

\vs<2->{Note: \alert{Weight functions} $\M{w}$ are \Blue{\bf not arbitrary
    functions}\\ -- need to be ``compatible'' with $\M{\gg}$.}

\bigskip
\vs<3->{Why? Compare for example $a$ and $f(a)$ with arbitrary $\gg$
  and $w$.}

\end{frame}


 % ---------------------------------------------------------
\begin{frame}{Weight Functions, Ground Case}

    A \alert{weight function} $\M{w: \Sigma \rightarrow \nat}$
    is any function satisfying:

    \begin{itemize}
    \item $w(a)>0$ for any constant $a\in\Sigma$;\\[.5em]
      
    \item<2-> if $w(f)=0$ for a unary function $f\in\Sigma$, then $f\gg g$
      for all functions $g\in\Sigma$ with $f\neq g$.\\

      That is, $f$ is the greatest element of $\Sigma$ wrt $\gg$.
      
     \end{itemize}

     \bigskip
\vs<3->{As a consequence, there is at most one unary function $f$ with
$w(f)=0$.}


\end{frame}


%-----------------------------------------

%

      \begin{frame}\frametitle{Exercise}

Consider a KBO ordering $\succ$ such that 
$inverse\gg times$ by precedence. 
Consider the literal: 
\[
  inverse(times(x,y) ) = times(inverse(y), inverse( x )).\]

Compare, w.r.t $\succ$, the left- and right-hand side terms of the
equality when: 

\begin{itemize}
\item $weight(inverse)=weigth(times)=1$;\bigskip\\[2em]


\item $weight(inverse) = 0$ and $weight(times) = 1$.

\end{itemize}

\end{frame}



%-----------------------------------------

%

      \begin{frame}\frametitle{Same Property as for $\BRiss$}

The conclusion is \alert{strictly smaller} than the rightmost premise:

 \M{\[
    \infer[(\ruleSup),]{
      s[r] \eql t \orl C \orl D
      }{
      \WildStrawberry{\underline{l \eql r}} \orl C
      &
      \WildStrawberry{\underline{s[l] \eql t}} \orl D
    }
    ~~~
    \infer[(\ruleSup),]{
      s[r] \neql t \orl C \orl D
      }{
      \WildStrawberry{\underline{l \eql r}} \orl C
      &
      \WildStrawberry{\underline{s[l] \neql t}} \orl D
    }
  \]}%
where (i) $\M{l \succ r}$, (ii) $\M{s[l] \succ t}$, (iii)
$\M{\M{l \eql r}}$ is strictly greater than any literal in $\M C$,
(iv) $\M{\M{s[l] \eql t}}$ is greater than or equal to
any literal in $\M D$.


                               \end{frame}

%---------------------------------------------------------------------


      \begin{frame}\frametitle{New redundancy}

Consider a superposition with a unit left premise:

 \[
    \infer[(\ruleSup),]{
      s[r] \eql t \orl D
      }{
      \underline{l \eql r}
      &
      \underline{s[l] \eql t} \orl D
    }
 \]

Note that we have

 \[
     l \eql r,
      s[r] \eql t \orl D
     \alert{\models}
      s[l] \eql t \orl D
\]

\vs<2->{
and we have

 \[
 s[l] \eql t \orl D
 \alert{\succ}
 s[r] \eql t \orl D.
 \]
}

\vs<3->{
If we also have $s[l] \eql t \orl D \succ l\eql r$, then the second
premise is \alert{redundant} and can be removed.
}

\medskip

\vs<4->{
  This rule (superposition plus deletion) is sometimes called
  \alert{demodulation} (also \alert{rewriting by unit equalities}).
}


                               \end{frame}



%---------------------------------------------------------------------
\begin{frame}
  \frametitle{Exercise}

  Consider the KBO ordering $\succ$ generated by:\\[.5em]

  -- the precedence $ f\gg a \gg b \gg c$;\\[.5em]

and\\[.5em]

-- the weight function $w$ with $w(f)=w(a)=w(b)=w(c)=1$.\\[1em]


Consider the set $S$ of ground formulas:
\begin{equation*}
  \begin{array}{l}
    a=b \orl a=c\\
    f(a)\neq f(b)\\
    b=c
    \end{array}
    \end{equation*}

   Apply saturation on S using an inference process based on the
   ground superposition calculus $\SUPiss$ (including the inference
   rules of ground binary resolution with selection).\\[.5em]

   Show that $S$ is unsatisfiable.
   \\[1em]
   
  \vs<2->{\alert{Challenge:} Show that $S$ is unsatisfiable such that during
    saturation \alert{only 4 new clauses} are generated.}
\end{frame}




\endinput
%---------------------------------------------------------------------
\begin{frame}
\frametitle{Exercise}
Consider the KBO ordering $\succ$ generated by:\\[.5em]

-- the precedence $ P\gg Q\gg f\gg a$;\\[.5em]

and\\[.5em]

-- the weight function $w$ with $w(P)=w(Q)=2$, $w(f)=w(a)=1$.\\[1em]


Consider the set of clauses $S$ to be: 
\[\begin{array}{l}
{Q(a)},\\
 {\neg Q(a)}\vee f(a)=a, \\ 
 {\neg P(a)},\\ {P(f(a))}\}.
\end{array}\] 

Apply saturation on $S$ by using an inferece process with redundancy
based on the (ground) superposition calculus $\SUPiss$. 
\end{frame}





