%---------------------------------------------------------------------

\begin{frame}
\frametitle{First-Order Logic  and TPTP}

\small
\begin{itemize}
\item \alert<1>{Language}: variables, function and predicate
  (relation) symbols. A {constant 
  symbol} is a special case of a function symbol.\\
  \vs<2->{\alert<2>{In TPTP}: {Variable names} start with upper-case letters.}
\item \vs<3->{\alert<3>{Terms}: variables, constants, and expressions
  $f(t_1,\ldots,t_n)$, where $f$ is a function symbol of arity $n$ and
  $t_1,\ldots,t_n$ are terms.} \vs<4->{Terms denote {\color<4>{blue}domain
  elements}.}
\item \vs<5->{\alert<5->{Atomic formula:} expression
      $p(t_1,\ldots,t_n)$, where $p$ is a predicate symbol of arity $n$
      and $t_1,\ldots,t_n$ are terms.} \vs<6->{Formulas denote
      {\color<6>{blue}properties of domain elements.}}
\item \vs<6->{All symbols are uninterpreted, apart from equality $=$.}
\end{itemize}
 
\smallskip

\begin{center}
\vs<7->{
\begin{tabular}{c|c}
FOL & TPTP \\
\hline
$\bot$, $\top$ & 
  \OliveGreen{\texttt{\$false}, \texttt{\$true}} \\ 
$\notl a$ &
   \OliveGreen{\texttt{\~{}a}} \\
$a_1 \andl \ldots \andl\ a_n$ &
  \OliveGreen{\texttt{a1 \& \ldots\ \& an}} \\
$a_1 \orl \ldots \orl\ a_n$ &
    \OliveGreen{\texttt{a1 | \ldots\ | an}} \\
$a_1 \implies a_2$ &
    \OliveGreen{\texttt{a1 => a2}} \\
$(\forall x_1)\ldots(\forall x_n)a$ &
   \OliveGreen{\texttt{! [X1,\ldots,Xn] : a}} \\
$(\exists x_1)\ldots(\exists x_n)a$ &
   \OliveGreen{\texttt{? [X1,\ldots,Xn] : a}}
\end{tabular}}
\end{center}

\end{frame}

%---------------------------------------------------------------------

\begin{frame}
  \frametitle{More on the TPTP Syntax}

\begin{itemize}
\item \vs<2->{{\color<2->{blue}Comments};}
\item \vs<3->{{\color<3->{Fuchsia}Input formula names};}
\item \vs<4->{\alert{Input formula roles} (very important);}
\item \vs<5->{\OrangeRed{Equality}}
\end{itemize}


\begin{small}
\begin{alltt}
{\color<2->{blue}\%----~1~*~x~{\color<5>{OrangeRed}=}~x}\\
fof({\color<3->{Fuchsia}left\_identity},\alert<4>{axiom},(\\
~~!~[X]~:~mult(e,X)~{\color<5>{OrangeRed}=}~X~)).\\
{\color<2->{blue}\%----~i(x)~*~x~{\color<5>{OrangeRed}=}~1}\\
fof({\color<3->{Fuchsia}left\_inverse},\alert<4>{axiom},(\\
~~!~[X]~:~mult(inverse(X),X)~{\color<5>{OrangeRed}=}~e~)).\\
{\color<2->{blue}\%----~(x~*~y)~*~z~{\color<5>{OrangeRed}=}~x~*~(y~*~z)}\\
fof({\color<3->{Fuchsia}associativity},\alert<4>{axiom},(\\
~~!~[X,Y,Z]~:\\
~~~~~~~mult(mult(X,Y),Z)~{\color<5>{OrangeRed}=}~mult(X,mult(Y,Z))~)).\\
{\color<2->{blue}\%----~x~*~x~{\color<5>{OrangeRed}=}~1}\\
fof({\color<3->{Fuchsia}group\_of\_order\_2},\alert<4>{hypothesis},\\
~~!~[X]~:~mult(X,X)~{\color<5>{OrangeRed}=}~e~).\\
{\color<2->{blue}\%----~prove~x~*~y~{\color<5>{OrangeRed}=}~y~*~x}\\
fof({\color<3->{Fuchsia}commutativity},\alert<4>{conjecture},\\
~~!~[X,Y]~:~mult(X,Y)~{\color<5>{OrangeRed}=}~mult(Y,X)~).
\end{alltt}
\end{small}

\end{frame}

%---------------------------------------------------------------------
%---------------------------------------------------------------------

\begin{frame}
  \frametitle{Proof by Vampire (Slightly Modified)}

\vspace*{-2ex}

\footnotesize
\begin{alltt}
\hskip-3em\alert<7>{Refutation found}.\\*[-0.7ex]
\hskip-3em\alert<6,7>{\alert<9>{270}. \$false [{\color<8>{blue}trivial inequality removal} 269]}\\*[-0.7ex]
\hskip-3em\alert<6>{\alert<9>{269}. mult(sk0,sk1) != mult (sk0,sk1) [{\alert<8>{superposition}} 14,125]}\\*[-0.7ex]
\hskip-3em\alert<6>{\alert<9>{125}. mult(X2,X3) = mult(X3,X2) [\alert<8>{superposition} 21,90]}\\*[-0.7ex]
\hskip-3em\alert<6>{\alert<9>{90}. mult(X4,mult(X3,X4)) = X3  [{\color<8>{blue}forward demodulation} 75,27]}\\*[-0.7ex]
\hskip-3em\alert<6>{\alert<9>{75}. mult(inverse(X3),e) = mult(X4,mult(X3,X4)) [\alert<8>{superposition} 22,19]}\\*[-0.7ex]
\hskip-3em\alert<2,6>{\alert<9>{27}. mult(inverse(X2),e) = X2 [\alert<8>{superposition} 21,11]}\\*[-0.7ex]
\hskip-3em\alert<6>{\alert<9>{22}. mult(inverse(X4),mult(X4,X5)) = X5 [{\color<8>{blue}forward demodulation} 17,10]}\\*[-0.7ex]
\hskip-3em\alert<2,6>{\alert<9>{21}. mult(X0,mult(X0,X1)) = X1 [{\color<8>{blue}forward demodulation} 15,10]}\\*[-0.7ex]
\hskip-3em\alert<6>{\alert<9>{19}. e = mult(X0,mult(X1,mult(X0,X1))) [\alert<8>{superposition} 12,13]}\\*[-0.7ex]
\hskip-3em\alert<6>{\alert<9>{17}. mult(e,X5) = mult(inverse(X4),mult(X4,X5)) [\alert<8>{superposition} 12,11]}\\*[-0.7ex]
\hskip-3em\alert<6>{\alert<9>{15}. mult(e,X1) = mult(X0,mult(X0,X1)) [\alert<8>{superposition} 12,13]}\\*[-0.7ex]
\hskip-3em\alert<4>{\alert<9>{14}. mult(sK0,sK1) != mult(sK1,sK0) [cnf transformation 9]}\\*[-0.7ex]
\hskip-3em\alert<4>{\alert<9>{13}. e = mult(X0,X0) [cnf transformation 4]}\\*[-0.7ex]
\hskip-3em\alert<4>{\alert<9>{12}. mult(X0,mult(X1,X2)) = mult(mult(X0,X1),X2) [cnf transformation 3]}\\*[-0.7ex]
\hskip-3em\alert<4>{\alert<9>{11}. e = mult(inverse(X0),X0) [cnf transformation 2]}\\*[-0.7ex]
\hskip-3em\alert<4>{\alert<9>{10}. mult(e,X0) = X0 [cnf transformation 1]}\\*[-0.7ex]
\hskip-3em\alert<2,4,5>{\alert<9>{9}. mult(sK0,sK1) != mult(sK1,sK0) [skolemisation 7,8]}\\*[-0.7ex]
\hskip-3em \alert<4,5>{\alert<9>{8}. ?[X0,X1]: mult(X0,X1) != mult(X1,X0) <=> mult(sK0,sK1) != mult(sK1,sK0) \\
\hskip33em[choice axiom]}\\*[-0.7ex]
\hskip-3em\alert<4>{\alert<9>{7}. ?[X0,X1]: mult(X0,X1) != mult(X1,X0) [ennf transformation 6]}\\*[-0.7ex]
\hskip-3em\alert<4,7>{\alert<9>{6}. \vampNot![X0,X1]: mult(X0,X1) = mult(X1,X0) [negated conjecture 5]}\\*[-0.7ex]
\hskip-3em\alert<3>{\alert<9>{5}. ![X0,X1]: mult(X0,X1) = mult(X1,X0) [input]}\\*[-0.7ex]
\hskip-3em\alert<3>{\alert<9>{4}. ![X0]: e = mult(X0,X0)[input]}\\*[-0.7ex]
\hskip-3em\alert<3>{\alert<9>{3}. ![X0,X1,X2]: mult(X0,mult(X1,X2)) = mult(mult(X0,X1),X2) [input]}\\*[-0.7ex]
\hskip-3em\alert<3>{\alert<9>{2}. ![X0]: e = mult(inverse(X0),X0) [input]}\\*[-0.7ex]
\hskip-3em\alert<3>{\alert<9>{1}. ![X0]: mult(e,X0) = X0 [input]}\\*[-0.7ex]
\end{alltt}

\vspace*{-3ex}


\begin{itemize}
\item \vs<2->{\alert<2>{Each inference derives a formula from zero or more other formulas;}}
\item \vs<3->{\alert<3>{Input}, \alert<4>{preprocessing},
    \alert<5>{new symbols introduction}, \alert<6>{superposition 
  calculus}}
\item \vs<7->{\alert<7>{Proof by refutation}, \alert<8>{generating}
    and
    {\color<8>{blue}simplifying} inferences,
  \alert<9>{unused formulas} \ldots}
\end{itemize}

                                \end{frame}

%---------------------------------------------------------------------


%---------------------------------------------------------------------



                      \begin{frame}
                 \frametitle{Vampire}


\begin{itemize}
\item \alert{Completely automatic:} once you started a proof attempt,
  it can only be interrupted by terminating the process.\\[1em]

%\item \visible<2->{Chris Weidenbach: \alert{A dark side of theorem proving.}}

%\item \visible<3->{Anonymous referee: \alert{\textsc{Vampire} is not
%a glass of Tuborg.}}
\item \visible<2->{\alert{Champion} of the CASC world-cup  in
    first-order theorem proving: won CASC $>$ 50
  times. \\[.5em]

\includegraphics[scale=.4]{vampire_trophies.jpg}


}
\end{itemize}

                               \end{frame}

%---------------------------------------------------------------------

                               \begin{frame}
              \frametitle{What an Automatic Theorem Prover
                is Expected to Do}

\OliveGreen{Input:}

\begin{itemize}
  \item a set of \Blue{axioms} (first order formulas) or clauses;

  \item a \Blue{conjecture} (first-order formula or set of clauses).
\end{itemize}

\OliveGreen{Output:}

\begin{itemize}
  \item \RedOrange{proof} (hopefully).
\end{itemize}

                               \end{frame}

%---------------------------------------------------------------------

                         \begin{frame}
                     \frametitle{Proof by Refutation}

Given a problem with axioms and assumptions $\xone{F}{n}$ and 
conjecture $G$,

\begin{enumerate}
\item negate the conjecture;
\item establish \alert{unsatisfiability} of the set of formulas
$\xone{F}{n},\notl G$.
\end{enumerate}

\bigskip

\visible<2->{
  Thus, we reduce the theorem proving problem to the problem of
  \alert{checking unsatisfiability}.
}

\bigskip

\visible<3->{
  In this formulation the negation of the conjecture $\notl G$ is
  treated like any other formula. In fact, Vampire (and other provers)
  \Fuchsia{internally treat conjectures differently, to make proof search more
  goal-oriented}. 
}


                                \end{frame}

%---------------------------------------------------------------------

                            \begin{frame}
                              \frametitle{General Scheme (simplified)}

\begin{itemize}
\item \alert<1>{Read a problem};
\item Determine \alert<1>{proof-search options} to be used for this problem;
\item \alert<1>{Preprocess the problem};
\item \alert<1>{Convert it into CNF};
\item {\color<2->{blue}Run a \alert{saturation algorithm} on it, try to derive $false$.}
\item If $false$ is derived, report the \alert<1>{result}, maybe including
  a refutation. 
\end{itemize}

\vs<2->{\color{blue}
  Trying to derive $false$ using a saturation algorithm is the \alert{hardest
  part}, which in practice may not terminate or run out of memory.
}

                                \end{frame}

