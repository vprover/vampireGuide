%---------------------------------------------------------------------
\renewcommand{\eql}{=}
                       \begin{frame}\frametitle{Subsumption and Tautology Deletion}

A clause is a propositional tautology if it is of the
form $\M{p \orl \notl p \orl C}$, that is, it contains a pair of
complementary literals.

There are also \Blue{equational tautologies}, for example
$\M{a \neql b \orl b \neql c \orl f(c,c) \eql f(a,a)}$. 

\medskip

\vs<2->{
  A clause $\M{C}$ \alert{subsumes} any clause $\M{C \orl D}$, where
  $D$ is non-empty.
}

\medskip

\vs<3->{
  It was known since 1965 that \RawSienna{subsumed clauses and propositional
    tautologies can be removed from the search space.} 
}


                                \end{frame}

%---------------------------------------------------------------------

                       \begin{frame}\frametitle{Problem}

How can we \alert{prove} that \Blue{completeness is preserved} if we 
\Fuchsia{remove subsumed clauses and tautologies} from the
\OliveGreen{search space}? 

\bigskip

\vs<2->{
  Solution: general \alert{theory of redundancy}.
}


                                \end{frame}
\endinput
%---------------------------------------------------------------------


                        \begin{frame}
\frametitle{Bag Extension of an Ordering}

\alert{Bag = finite multiset}.

Let $\M{>}$ be any (strict) ordering on a set $\M{X}$. 
The \alert{bag extension of $>$} is a binary relation \alert{$\mextension{>}$}, 
on bags over $\M{X}$, defined as the smallest transitive relation on 
bags such that
  \[
  \begin{array}{l}
    \setof{x,\xone{y}{n}} \mextension{>}
    \setof{\xone{x}{m},\xone{y}{n}} \\
    \qquad \text{if } x > x_i \text{ for all } i\in\{1\ldots m\},
    \end{array}
  \]
where $\M{m \geq 0}$. 

\vs<2->{
  \alert{Idea:} a bag becomes smaller if we replace
  an element by \Blue{any finite number} of smaller elements.
}

\vs<3->{
The following \alert{results are known} about the bag extensions 
of orderings:

  \begin{enumerate}
  \item $\M{\mextension{>}}$ is an \OliveGreen{ordering};
  \item If $\M{>}$ is \OliveGreen{total}, then so is $\M{\mextension{>}}$;
  \item If $\M{>}$ is \OliveGreen{well-founded}, then so is $\M{\mextension{>}}$.
  \end{enumerate}
}

                                \end{frame}

%---------------------------------------------------------------------

                        \begin{frame}\frametitle{Clause Orderings}

From now on consider clauses also as \alert{bags of literals}. Note:

\begin{itemize}
\item we have an ordering $\succ$ for comparing literals;
\item a clause is a bag of literals.
\end{itemize}

\vs<2->{
Hence

\begin{itemize}
\item we can compare clauses using the \alert{bag extension $\succm$} of
      $\M{\succ}$.
\end{itemize}
}

\vs<3->{
  For simpicity we denote the multiset ordering also by
  $\alert{\succ}$.
}

                                \end{frame}

%---------------------------------------------------------------------
\begin{frame}\frametitle{Example}

  Let $\succ$ be a total well-founded ordering on the ground atoms
$p_1, \ldots, p_6$ such that $p_6\succ p_5\succ p_4\succ p_3 \succ p_2
\succ p_1$. Consider
the bag extension of $\succ$; for simplicity, denote the bag extension 
of $\succ$ also by $\succ$. 

\bigskip


Using $\succ$, compare and order the following three clauses: 
\[p_6\vee \neg p_6,\qquad \neg p_2\vee
  p_4\vee p_5, \qquad p_2\vee p_3.\]

 \end{frame} 
                                
% ---------------------------------------------------------------------
                                
                     \begin{frame}\frametitle{Redundancy}

A clause $C \in S$ is called \alert{redundant in $S$} if it is a
logical consequence of clauses in $S$ strictly smaller than $C$.

                              \end{frame}


%---------------------------------------------------------------------

              	   \begin{frame}
           \frametitle{Examples}

A \alert{tautology} $p \orl \notl p \orl C$ is a logical consequence of the
empty set of formulas:

\[
\models p \orl \notl p \orl C,
\]
therefore it is \OliveGreen{redundant}.

\vs<2->{
We know that \alert{$C$ subsumes $C \orl D$}. Note

\[
  \begin{array}{l}
    C \orl D \succ C\\
    C \models C \orl D
 \end{array}
\]
therefore subsumed clauses are \OliveGreen{redundant}. 
}

\medskip

\vs<3>{
  If $\emptyclause \in S$, then all non-empty other clauses in $S$ are
  \OliveGreen{redundant}. 
}

                           \end{frame}

                
%---------------------------------------------------------------------
\begin{frame}
  \frametitle{Redundant Clauses Can be Removed}


  In $\BRiss$ (and in all calculi we will consider later) 
  \Blue{\alert<2>{redundant clauses can be removed from the search space}}.

\end{frame}
%---------------------------------------------------------------------
\end{document}

                \begin{frame}\frametitle{Inference Process with Redundancy}

Let $\M{\isI}$ be an inference system. Consider an inference process with
two kinds of step $\M{S_i \RR S_{i+1}}$:

\begin{enumerate}
\item \Blue{Adding the conclusion} of an $\M{\isI}$-inference with
  premises in $S_i$.

\item \Blue{Deletion of a clause redundant} in $\M{S_i}$, that is 

  \[\M{S_{i+1} = S_i - \setof{C}},\]
where $\M{C}$ is redundant in $\M{S_i}$.
\end{enumerate}

                                \end{frame}

%---------------------------------------------------------------------


                 \begin{frame}\frametitle{Fairness: Persistent Clauses and Limit}

Consider an inference process

\[\M{
  S_0 \RR S_1 \RR S_2 \RR \ldots
}\]

A clause $\M{C}$ is called \alert{persistent} if

\[\M{
    \exists i \forall j \geq i (C \in S_j).
}\]
The \alert{limit $S_{\omega}$} of the inference process is
the set of all persistent clauses:

\[\M{
  S_\omega = \bigcup_{i=0,1,\ldots}\bigcap_{j \geq i} S_j.
}\]

                                \end{frame}

%---------------------------------------------------------------------


                            \begin{frame}\frametitle{Fairness}

The process is called \alert{$\isI$-fair} if every inference with persistent
premises in $\M{S_\omega}$ has been applied, that is, if 

    \[\M{
      \infer{C}{C_1 & \ldots & C_n}}
    \]
is an inference in $\M{\isI}$ 
and $\M{\setof{\xone{C}{n}} \subseteq S_\omega}$,
then $\M{C \in S_i}$ for some $\M{i}$.


                                \end{frame}

%---------------------------------------------------------------------

                    \begin{frame}\frametitle{Completeness of $\BRiss$}


\textbf{\OliveGreen{Completeness Theorem.}}
Let $\M{\succ}$ be a well-founded ordering and $\M{\sel}$ a
well-behaved selection function. Let also 

\begin{enumerate}
\item $\M{S_0}$ be a set of clauses;
\item $\M{S_0 \RR S_1 \RR S_2 \RR \ldots}$ be a fair 
$\M{\BRiss}$-inference process.
\end{enumerate}
Then $\M{S_0}$ is unsatisfiable if and only if $\M{\emptyclause \in S_i}$
for some $\M{i}$.

                                \end{frame}

%---------------------------------------------------------------------


                      \begin{frame}\frametitle{Saturation up to Redundancy}

A set $\M{S}$ of clauses is called \alert{saturated up to redundancy} if
for every $\M{\isI}$-inference 

    \[\M{
      \infer{C}{C_1 & \ldots & C_n}}
    \]
with premises in $\M{S}$, either

\begin{enumerate}
\item $\M{C \in S}$; or
\item $\M{C}$ is redundant w.r.t.\ $\M{S}$, that is, 
$\M{S_{\prec C} \models C}$.
\end{enumerate}

                                \end{frame}

%---------------------------------------------------------------------

\begin{frame}
\frametitle{Saturation up to Redundancy and Satisfiability Checking}

\textbf{\OliveGreen{Lemma.}} A set $\M S$ of clauses saturated up to redundancy
is unsatisfiable if and only if $\M{\emptyclause \in S}$.

\bigskip

\vs<2->{
  Therefore, if we built a set saturated up to redundancy,
  then the initial set $S_0$ is \alert{satisfiable}. This is a powerful
  way of checking redundancy: one can even check satisfiability of
  formulas having only \Blue{infinite models}.
}

\bigskip

\vs<3->{
  The only problem with this characterisation is that there is \Purple{no obvious way
  to build a model of $S_0$} out of a saturated set.
}


                                \end{frame}


%---------------------------------------------------------------------

              	   \begin{frame}
           \frametitle{Binary Resolution with Selection}

One of the \alert{key properties} to satisfy this lemma is the following:
\Purple{the conclusion of every rule is strictly smaller that the
rightmost premise of this rule.}

\begin{itemize}
\item
  \Blue{Binary resolution},

  \[
      \infer[(\BRr).]{C_1 \orl C_2}{\underline{p} \orl C_1 & 
                                    \underline{\notl p} \orl C_2}
  \]

\item
\Blue{Positive factoring},

  \[
    \begin{array}[b]{l}
      \infer[(\Fact).]{p \orl C}{\underline{p} \orl \underline{p} \orl C}
    \end{array}
  \]
\end{itemize}


                                \end{frame}

%---------------------------------------------------------------------
